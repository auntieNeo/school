\documentclass[12pt, letterpaper]{article}
\usepackage{ifpdf}
\usepackage{mla}
\begin{document}
\begin{mla}{Jonathan}{Glines}{Professor Pelletti}{Philosophy 1101}{28 September 2011}{First and Second Meditations of Descarte's \textit{Meditations on First Philosophy} and their Implications Concerning the External World}

%Possible Topics for Essay 2
%
%The final essay should be from 4 to 6 pages long and address a subject directly relating to the course. Do not use non-peer reviewed material from the Internet, but you may use the Library resources. Any outside sources must be included in a bibliography. The best strategy, however, is simply to concentrate on the material in your text. To receive a good grade you must show a strong understanding of the course material. Make sure that you have a clear thesis. 
%
%1) Can we beyond any doubt demonstrate the existence of the external world? Should we doubt its existence? Why or why not? (Besides Descartes, I recommend reading Bertrand Russell, The Problems of Philosophy, Chapter II. Note, this topic is not asking whether or not the external world exists.)
%
%2) According to Descartes, where do our errors come from and how can we avoid them? (Take into account Meditation 4 and do not confuse the verb ‘to err’ with the noun ‘error.’ The Cartesian method might assist you in the second part of this topic.)
%
%3) What is the function and justification of Cartesian doubt?
%
%4) Is Descartes successful in demonstrating God’s existence? (Be careful to distinguish the truth or falsity of the conclusion, or your belief in the conclusion’s truth of falsity, from the validity of the argument. You may look at either the 3rd or the 5th Meditation, if the 5th consider as well Anselm and Gaunilo. )
%
%5) According to Descartes, what is the proper role of the senses? How should they not be used? (This topic is not assisted by an enumeration of the senses.)
%
%6) Contrast Plato’s view of concepts with Wittgenstein’s, taking a definite position with regard to them.
%
%7) Argue for or against Mill’s Liberty Principle, considering as well its basis.
%
%8) Compare Hume’s views on miracles with those of Leibniz. Is a reconciliation of their views possible?

% I think I'm going to do number 1 because that sees to be the easiest thing to talk about.

% I. Introduction
%   A. Thesis Statement
%     1. The way Descarte explains in his 1st meditation, it seems clear that we cannot eliminate all doubt outside of our mind.
%     2. This does not mean we should doubt existance of the external world outright, because this only demonstrates that it can be doubted, not that it doesn't exist.
%     3. This conclusion is not useful as 

In the first meditation, Descarte considers the opinions and prejudices that he has taken for granted througouht his life. Knowing that many of them have proven to be false, he attempts to remedy this uncertainty by disregarding all of his opinions that carry even the slightest doubt, such that he might find something certain that cannot be doubted. His observation that an evil demon might have been manipulating his perceptions all of his life, as though showing him some elaborate dream, casts doubt on everything in the external world. The only thing Descarte can know with absolute certainty is that he exists as a thinking thing, so long as he is thinking that he exists.

It is clear by this point that anything and everything in the external world which we percieve can be doubted, so long as we accept that we could be dreaming or might be being decieved by an evil demon. That much is clear and logically consistant, but there is an important distinction to make. In the first meditation, Descarte aims to avoid all fallacy by simply rejecting opinions that can be doubted at all.
\begin{mlaquote}
``But reason now persuades me that I should withold my assent no less carefully from opinions that are not completely certain and indubitable than I would from those that are patently false. For this reason, it will suffice for the rejection of all these opinions, if I find in each of them some reason for doubt."
\end{mlaquote}
It is a mistake to think that by temporaraly rejecting the external world as doubtable that Descarte has proven that it does not exist. Descarte has only proven that it can be doubted, which is a nessecary condition for the external world not existing, but not a sufficient condition for it not existing. In other words, the external world not existing implies that it can be doubted, the latter of which Descarte has proved. But to say that the external world being doubtable implies that it does not exist would be a fallacy at this point in the meditation, because our mere doubt of existance of an external world tells us nothing, since ultimately we may have misplaced this (itself dubious) doubt.

With these things in mind, it might seem like Descarte has accomplished little more than doubt and doubt upon that doubt in his first meditation. ``It is as if I had suddenly fallen into a deep whirlpool; I am so tossed about that I can neither touch bottom with my foot, nor swim up to the top." In the second meditation, he continues his quest for absolute certainty, and stumbles upon a gem of certainty that he cannot doubt, and his famous assertion: that I (and, I can only hope, Descarte and my reader) am certain that I exist, so long as I am thinking that I exist. At first it seems that this is a profound discovery and a relief from the disorientating doubt of the first meditation. But this is another rather vacuous conclusion, because this says nothing of the existance or non-existance of anything outside of my mind.

If we are careful to see the vacuousness of our affirmations so far in these meditations, the tendency is not to fall into inescapable doubt and skepticism, but to call into question what, if anything, we have accomplished by meditating with Descarte. Bertrand Russel provides a possible answer to this question in chapter 2 of \textit{The Problems of Philosophy}, suggesting that the self is not the only thing that we can know for certain. The sensations that we experiance, independant of the existance of the external world, are just as certain. Russel calls these sensations sense-data, and writes,
\begin{mlaquote}``Although we are doubting the physical existence of the table, we are not doubting the existence of the sense-data which made us think there was a table; we are not doubting that, while we look, a certain color and shape appear to us, and while we press, a certain sensation of hardness is experienced by us. All this, which is phychological, we are not calling in question."\end{mlaquote}
This observation does not prove Descartes evil demon scenario to be false, and Russel concedes to this point. However, it does extend the boundry on our realm of certainty past the boundry established in the second meditation. In addition to this, perhaps more relevant to our everyday lives, this line of thought gives us a new way of thinking about our perceptions and how they relate to the external world, regardless of its uncertainty. Towards the end of chapter 2, Russel tries to use our certainty of the exisentance of sense-data, along with the common-sense rationalization that there must be a common object behind these sense-data 



%\begin{workscited}
%
%\bibent
%Voltaire.  \textit{Candide}.  Trans. Philip Littell.  New York: Boni and Liverlight, Inc., 1918.  Print.
%
%\end{workscited}
\end{mla}
\end{document}
