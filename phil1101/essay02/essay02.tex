\documentclass[12pt, letterpaper]{article}
\usepackage{ifpdf}
\usepackage{mla}
\begin{document}
\begin{mla}{Jonathan}{Glines}{Professor Pelletti}{Philosophy 1101}{28 September 2011}{Justifying Evil in a Perfect World on Gottfried Leibniz's \textit{Discourse on Metaphysics}}

%Possible Topics for Essay 2
%
%The final essay should be from 4 to 6 pages long and address a subject directly relating to the course. Do not use non-peer reviewed material from the Internet, but you may use the Library resources. Any outside sources must be included in a bibliography. The best strategy, however, is simply to concentrate on the material in your text. To receive a good grade you must show a strong understanding of the course material. Make sure that you have a clear thesis. 
%
%1) Can we beyond any doubt demonstrate the existence of the external world? Should we doubt its existence? Why or why not? (Besides Descartes, I recommend reading Bertrand Russell, The Problems of Philosophy, Chapter II. Note, this topic is not asking whether or not the external world exists.)
%
%2) According to Descartes, where do our errors come from and how can we avoid them? (Take into account Meditation 4 and do not confuse the verb ‘to err’ with the noun ‘error.’ The Cartesian method might assist you in the second part of this topic.)
%
%3) What is the function and justification of Cartesian doubt?
%
%4) Is Descartes successful in demonstrating God’s existence? (Be careful to distinguish the truth or falsity of the conclusion, or your belief in the conclusion’s truth of falsity, from the validity of the argument. You may look at either the 3rd or the 5th Meditation, if the 5th consider as well Anselm and Gaunilo. )
%
%5) According to Descartes, what is the proper role of the senses? How should they not be used? (This topic is not assisted by an enumeration of the senses.)
%
%6) Contrast Plato’s view of concepts with Wittgenstein’s, taking a definite position with regard to them.
%
%7) Argue for or against Mill’s Liberty Principle, considering as well its basis.
%
%8) Compare Hume’s views on miracles with those of Leibniz. Is a reconciliation of their views possible?

\begin{workscited}

\bibent
Voltaire.  \textit{Candide}.  Trans. Philip Littell.  New York: Boni and Liverlight, Inc., 1918.  Print.

\end{workscited}
\end{mla}
\end{document}
