\documentclass[12pt]{article}
\special{papersize=8.5in,11in}
\usepackage[utf8]{inputenc}
\begin{document}
\section*{Topic: Machine Network Simulation in a Portable Library}
Our goal is to develop a portable library for quickly and easily instantiating many machine simulations on a virtual machine network. We will take both OS/161 and SYS/161 and package them into a library, written in portable ANSI C wherever possible for easy inclusion in application programs. This will allow for applications in education, research, and even entertainment.
\subsection*{Machine Network Optimized for Portability (libmnop)}
libmnop is the library that will realize this goal. A tentative design for libmnop is outlined below in three points. The outline is organized such that each point can be developed independently from the others.

Due to the scope of the design and the constraints of a semester, we intend to focus on developing point 1 while merely describing points 2 and 3 in the term report.
\subsubsection*{Design Outline}
\begin{enumerate}
\item[1] Portable implementations of machine simulators
\begin{enumerate}
\item[a] Expose API for creating and interfacing machines to the application programmer
\item[b] Use libtool to dynamically load machine implementations in a portable manner
\item[c] Provide tools to statically link machine implementations for maximum portability (i.e. without dynamically loading code)
\end{enumerate}
\item[2] Self-contained file format for storing machine states
\begin{enumerate}
\item[a] Allow for disparate implementations of machine state storage, but expose a consistent API to the application programmer
\item[b] Define file formats for storing machine states, including kernels, memory, disks, and peripheral states
\item[c] Exploit solid compression schemes to store many similar but slightly differing machine states efficiently
\end{enumerate}
\item[3] Simulating machine networks
\begin{enumerate}
\item[a] Expose API for creating and configuring networks to the application programmer
\item[b] Define machine communication interfaces suitable for describing the most common network technologies
\item[c] Define interface for communication outside of the library (i.e. communication from client machines to host machine and vice versa)
\end{enumerate}
\end{enumerate}

\subsection*{Similar Projects}
The design goals for libmnop are unconventional. There are existing multi-simulators and even network simulators, but few of these are designed to be used as a library.

However, there are a number of existing projects with similar goals or features. The following is a short list of such projects. Further research into these projects should motivate the development of libmnop.
\begin{itemize}
\item[{\tt 0x10}$^\textrm{{\tt c}}$] Sandbox video game that incorporates simulated computers as a game element. Developed by Markus ``Notch" Persson, the creator of Minecraft.
\item[GNS3] Graphical Network Simulator, open source application that allows for simulation of complex computer networks.
\item[libretro] A library developed for RetroArch that defines the API for emulation cores. This greatly improves the portability of game console emulators.
\item[libz80] A portable ANSI C library that emulates a Z80 processor.
\item[MAME/MESS] A popular multi-system emulator suite for emulating arcade systems.
\end{itemize}
\end{document}
