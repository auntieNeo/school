\documentclass[12pt]{article}
\special{papersize=8.5in,11in}
\usepackage[utf8]{inputenc}
\usepackage{amssymb,amsmath,amsthm}
\newtheorem*{thm}{Theorem}
\newtheorem{lem}{Lemma}
\usepackage{colonequals}
\usepackage{parskip}
\pagestyle{plain}
\begin{document}
\begin{flushright}
\Large{Jonathan Glines}
\end{flushright}
\section*{Section 2.4 Exercise 5}

\subsection*{Exercise 5}
Let $S$ be a set of nonnegative real numbers that is bounded above and let $T \colonequals \left\{x^2 : x \in S\right\}$. Prove that if $u = \sup S$, then $u^2 = \sup T$. Give an example that shows the conclusion may be false if the restriction against negative numbers is removed.


\subsubsection*{Solution}
\begin{thm}
Let $S$ be a set of nonnegative real numbers that is bounded above and let $T \colonequals \left\{x^2 : x \in S\right\}$. Prove that if $u = \sup S$, then $u^2 = \sup T$.
\end{thm}

Let $x$ be any element in $T$. Since $\sqrt{x} \leq x$ and $x$ is bounded above, it is clear that $\sqrt{x}$ is also bounded above. Let $U \colonequals \left\{\sqrt{x} : x \in T\right\}$.

If we modify this theorem and let $S \colonequals \left\{x : -2 \leq x \leq 0\right\}$ contain negative numbers, then $\left(\sup T\right)^2 = 0 \neq \sup T$ and the conclusion is no longer true.

\section*{Section 2.5 Exercises 1 and 7}

\subsection*{Exercise 1}
If $I \colonequals \left[a, b\right]$ and $I' \colonequals \left[a', b'\right]$ are closed intervals in $\mathbb{R}$, show that $I \subseteq I'$ if and only if $a' \leq a$ and $b \leq b'$.

\subsubsection*{Solution}
To show that $I'$ is is a subset of $I$, it is sufficient to show that the endpoints $a$ and $b$ are contained in $I'$. Knowing that $a \leq b$, we can write the inequality $a' \leq a \leq b \leq b'$. This clearly shows that $a$ and $b$ are contained in $I'$, so by Theorem 2.5.1 (Characterization Theorem of Intervals) FIXME

\subsection*{Exercise 7}
Let $I_n \colonequals \left[0, 1/n\right]$ for $n \in \mathbb{N}$. Prove that $\bigcap_{n = 1}^\infty I_n = \left\{0\right\}$.

\subsubsection*{Solution}
From the definition of the interval $I_n$, it is obvious that for any $x \in I_n$, $x \geq 0$.

Let $y$ be any element in $I_n$ such that $y > 0$. FIXME: need to prove there exists a number $z = 1/n$ with $n \in \mathbb{N}$ such that $z < y$.

\section*{Section 3.1 Exercises 5(b)(d) and 12}

\subsection*{Exercise 5(b)(d)}
Use the definition of a sequence to establish the following limits.
\begin{itemize}
\item[(b)]$\displaystyle\lim\left(\frac{2n}{n + 1}\right) = 2$
\item[(d)]$\displaystyle\lim\left(\frac{n^2 - 2}{2n^2 + 3}\right) = \frac{1}{2}$
\end{itemize}

\subsubsection*{Solution}
\begin{itemize}
\item[(b)]
Given $\varepsilon > 0$, we want to find $K \in \mathbb{N}$ such that
\begin{equation*}
n \geq K \implies \left|\frac{2n}{n + 1} - 2\right| < \varepsilon \text{.}
\end{equation*}
We can simplify this inequality as follows:
\begin{align*}
\left|\frac{2n}{n + 1} - 2\right| &= \left|\frac{2n - 2n - 2}{n + 1}\right| = \left|\frac{-2}{n + 1}\right| = \frac{2}{n + 1} < \varepsilon\text{.}\\
\end{align*}
Since $\frac{2}{n + 1} < \frac{2}{n}$

\begin{align*}
\lim\left(\frac{2n}{n + 1}\right) &= 2\\
&= \lim\left(\frac{2n}{n + 1} \cdot \frac{1/n}{1/n}\right) = \lim\left(\frac{2}{1 + 1/n}\right)
\end{align*}

\item[(d)]$\displaystyle\lim\left(\frac{n^2 - 2}{2n^2 + 3}\right) = \frac{1}{2}$

\end{itemize}

\subsection*{Exercise 12}
Show that $\displaystyle\lim\left(\frac{1}{n} - \frac{1}{n + 1}\right) = 0$.

\section*{Section 3.2 Exercises 2, 4, 7, and 15}

\subsection*{Exercise 2}
Give an example of two divergent sequences $X$ and $Y$ such that:
\begin{itemize}
\item[(a)]
their sum $X + Y$ converges,
\item[(b)]
their product $XY$ converges.
\end{itemize}

\subsection*{Exercise 4}
Show that if $X$ and $Y$ are sequences such that $X$ converges to $x \neq 0$ and $XY$ converges, then $Y$ converges.

\subsection*{Exercise 7}
If $\left(b_n\right)$ is a bounded sequence and $\lim\left(a_n\right) = 0$, show that $\lim\left(a_n b_n\right) = 0$. Explain why Theorem 3.2.3 \textit{cannot} be used.

\subsection*{Exercise 15}
Show that if $z_n \colonequals \left(a^n + b^n\right)^{1/n}$ where $0 < a < b$ , then $\lim\left(z_n\right) = b$.

\section*{Section 3.3 Exercise 1}

\subsection*{Exercise 1}
Let $x_1 \colonequals 8$ and $x_{n + 1} \colonequals \frac{1}{2} x_n + 2$ for $n \in \mathbb{N}$. Show that $\left(x_n\right)$ is bounded and monotone. Find the limit.

\end{document}
