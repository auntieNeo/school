\documentclass[12pt]{article}
\special{papersize=8.5in,11in}
\usepackage[utf8]{inputenc}
\usepackage{amssymb,amsmath}
\pagestyle{plain}
\begin{document}
\begin{flushright}
\Large{Jonathan Glines}
\end{flushright}
\begin{flushleft}
\section*{Section 1.1 Exercises 2, 14, 16, 19(b), 22(b)}
\subsection*{Exercise 2} Draw diagrams to simplify and identify the following sets:
\begin{enumerate}
\item[(a)] $A\setminus\left(B \setminus A\right)$
\newline
\newline
\newline
\newline
\newline
\newline
\newline
\item[(b)] $A\setminus\left(A \setminus B\right)$
\newline
\newline
\newline
\newline
\newline
\newline
\newline
\item[(c)] $A \cap \left(B \setminus A\right)$
\newline
\newline
\newline
\newline
\newline
\newline
\newline
\end{enumerate}

\subsection*{Exercise 14}
Show that if $f : A \rightarrow B$ and $E, F$ are subsets of $A$, then $f\left(E \cup F\right) = f\left(E\right) \cup f \left(F\right)$ and $f\left(E \cap F\right) \subseteq f\left(E\right)\cap f\left(F\right)$.
\subsubsection*{Solution}
If $G = f\left( E \cup F \right)$, then $E$ and $F$ are subsets of the inverse image of $G$ under $f$. Because of this, the set $f\left(E\right)$ contains a subset of the elements in $G$, as does the set $f\left(F\right)$. Together, $f\left(E\right)$ and $f\left(F\right)$ form a subset of $G$.

Because $E$ and $F$ together equal $E \cup F$, together they exhaust all of the inputs to $f$ that form $G$. Thus, $G$ is a subset of the union of $f\left(E\right)$ and $f\left(F\right)$, and $f\left(E \cup F\right) = f\left(E\right) \cup \left(F\right)$.

Because $E \cap F$ is a subset of $E \cup F$, it follows that $f\left(E \cap F\right)$ is a subset of $f\left(E \cup F\right)$.

\subsection*{Exercise 16}
Show that the function $f$ defined by $f\left(x\right) \mathrel{\mathop:}= x / \sqrt{x^2 + 1}, x \in \mathbb{R}$, is a bijection of $\mathbb{R}$ onto $\left\{y : -1 < y < 1\right\}$.
\subsubsection*{Solution}
Because $\sqrt{x^2 + 1} > \left|x\right|$, it follows that the absolute value of $x / \sqrt{x^2 + 1}$ must be less than $1$, and that $f$ maps \textbf{into} $\left\{y : -1 < y < 1\right\}$.


%Let $x_1, x_2$ be numbers in $\mathbb{R}$ such that $f(x_1) = f(x_2)$. It follows that,

\begin{align*}
\frac{x_1}{\sqrt{\left(x_1\right)^2 + 1}} &= \frac{x_2}{\sqrt{\left(x_2\right)^2 + 1}} \\
x_2\sqrt{\left(x_1\right)^2 + 1} &= x_1\sqrt{\left(x_2\right)^2 + 1} \\
\intertext{For $x_1, x_2 \geq 0$,}
\sqrt{\left(x_2\right)^2\left(\left(x_1\right)^2 + 1\right)} &= \sqrt{\left(x_1\right)^2\left(\left(x_2\right)^2 + 1\right)} \\
\left(x_2\right)^2\left(\left(x_1\right)^2 + 1\right) &= \left(x_1\right)^2\left(\left(x_2\right)^2 + 1\right) \\
\left(x_1 x_2\right)^2 + \left(x_2\right)^2 &= \left(x_1 x_2\right)^2 + \left(x_1\right)^2 \\
x_2 &= x_1
\end{align*}
Because $f\left(x_1\right) = f\left(x_2\right)$ is true if and only if $x_1 = x_2$, $f$ is \textbf{injective}.

Let $y_1$ be any number in $\left\{y : -1 < y < 1\right\}$. Suppose that $f(x) = y_1$.
\begin{align*}
y_1 &= \frac{x}{\sqrt{x^2 + 1}} \\
%y_1 &= \sqrt{\frac{x^2}{x^2 + 1}} \\
%\left(y_1\right)^2 &= \frac{x^2}{x^2 + 1} \\
%\frac{1}{\left(y_1\right)^2} &= 1 + \frac{1}{x^2} \\
%\frac{1}{\left(y_1\right)^2} - 1 &= \frac{1}{x^2} \\
%%\frac{1 - \left(y_1\right)^2}{\left(y_1\right)^2} &= \frac{1}{x^2} \\
%\frac{\left(y_1\right)^2}{1 - \left(y_1\right)^2} &= x^2 \\
%\sqrt{\frac{\left(y_1\right)^2}{1 - \left(y_1\right)^2}} &= x \\
\intertext{Using algebra, we can find $x$ in terms of $y_1$.}
\frac{y_1}{\sqrt{1 - \left(y_1\right)^2}} &= x \\
\intertext{We check that }
f\left(\frac{y_1}{\sqrt{1 - \left(y_1\right)^2}}\right) = y &= \frac{\frac{y_1}{\sqrt{1 - \left(y_1\right)^2}}}{\left(\frac{y_1}{\sqrt{1 - \left(y_1\right)^2}}\right)^2 + 1} \\
y &= \frac{y_1}{\sqrt{1 - \left(y_1\right)^2}\left(\frac{\left(y_1\right)^2}{1 - \left(y_1\right)^2} + 1\right)} \\
y &= \frac{y_1}{\sqrt{1 - \left(y_1\right)^2}\left(\frac{\left(y_1\right)^2 + 1 - \left(y_1\right)^2}{1 - \left(y_1\right)^2}\right)} \\
y &= \frac{y_1}{\frac{\sqrt{1 - \left(y_1\right)^2}}{1 - \left(y_1\right)^2}} \\
y &= \frac{y_1\left(1 - \left(y_1\right)^2\right)}{\sqrt{1 - \left(y_1\right)^2}} \\
y &= \frac{y_1 - \left(y_1\right)^3}{\sqrt{1 - \left(y_1\right)^2}} \\
y^2 &= \frac{\left(y_1 - \left(y_1\right)^3\right)^2}{1 - \left(y_1\right)^2} \\
y^2\left(1 - \left(y_1\right)^2\right) &= \left(y_1 - \left(y_1\right)^3\right)^2 \\
\end{align*}
Since we can express $x$ in terms of any $y_1$, $f$ must be \textbf{surjective}. Because $f$ is both injective and surjective, $f$ it is \textbf{bijective} by definition.

\subsection*{Exercise 19(b)}
Show that if $f : A \rightarrow B$ is surjective and $H \subseteq B$, then $f\left(f^{-1}\left(H\right)\right) = H$. Give an example to show that equality need not hold if $f$ is not surjective.
\subsubsection*{Solution}

The inverse image of $H$ under $f$ is by definition the set of all of the inputs to $f$ that produce a result that is in the set $H$. Because the set $f\left(f^{-1}\left(H\right)\right)$ is the direct image under the same $f$ of this inverse image, all of its elements must be in $H$.

Because $f$ is given to be surjective onto $B$, and $H$ is a subset $B$, for each $y \in H$ there must exist $x \in f^{-1}\left(H\right)$ such that $f\left(x\right) = y$. Knowing this, it's clear that the direct image of the set $f^{-1}\left(H\right)$ under $f$ will contain every element in $H$.

Because we know that $f\left(f^{-1}\left(H\right)\right) \subseteq H$ and that $f\left(f^{-1}\left(H\right)\right) \supseteq H$, we can conclude that $f\left(f^{-1}\left(H\right)\right) = H$.
\subsection*{Exercise 22(b)}
Show that if $g \circ f$ is surjective, then $g$ is surjective.
\subsubsection*{Solution}
Since $g \circ f$ is given to be surjective, its range contains the entire codomain. Because $g \circ f$ is a composition of functions, its range must be a subset of the range of $g$. Conversely, the range of $g$ is a superset of the range of $g \circ f$. Because $R\left(g\right) \supseteq R\left(g \circ f\right)$ and $R\left(g \circ f\right)$ equals the entire codomain, $R\left(g\right)$ must also equal the entire codomain and $g$ must be surjective.

\section*{Section 1.2 Exercises 9, 10, and 20}
\subsection*{Exercise 9}
Prove that $n^3 + \left(n + 1\right)^3 + \left(n + 2\right)^3$ is divisible by $9$ for all $n \in \mathbb{N}$.
\subsubsection*{Proof}
Let $S$ be the the set of all numbers $n$ such that $n^3 + \left(n + 1\right)^3 + \left(n + 2\right)^3 = 9k$ for some natural number $k$.

It can be shown that $1 \in S$ because the equation $1^3 + \left(1 + 1\right)^3 + \left(1 + 2\right)^3 = 36 = 9k$ is true for $k = 4$.

Given any number $n_1 \in S$, it can be shown that,
\begin{align*}
\left(n_1\right)^3 + \left(n_1 + 1\right)^3 + \left(n_1 + 2\right)^3 &= 9k \\
\left(n_1 + 1\right)^3 + \left(n_1 + 2\right)^3  + \left(n_1 + 3\right)^3 &= \left(n_1 + 1\right)^3 + \left(n_1 + 2\right)^3 + \left(n_1\right)^3 + 9\left(n_1\right)^2 + 27n_1 + 27\\
&= 9k + 9\left(n_1\right)^2 + 27n_1 + 27 \\
&= 9\left(k + \left(n_1\right)^2 + 3n_1 + 3\right)\text{,} \\
\end{align*}
which means that $n_1 + 1 = n_2$ satisfies the equation $\left(n_2\right)^3 + \left(n_2 + 1\right)^3 + \left(n_2 + 2\right)^3 = 9k_2$ for $k_2 = k + \left(n_1\right)^2 + 3n_1 + 3$. This tells us that $n \in S$ implies $n + 1 \in S$, so by mathematical induction $S = \mathbb{N}$ and $n^3 + \left(n + 1\right)^3 + \left(n + 2\right)^3$ is divisible by 9 for all $n \in \mathbb{N}$.

\subsection*{Exercise 10}
Conjecture a formula for the sum $1 / 1 \cdot 3 + 1 / 3 \cdot 5 + \cdots + 1 / \left(2n - 1 \right)\left(2n + 1\right)$, and prove your conjecture by using Mathematical Induction.
\subsection*{Exercise 20}
Let the numbers $x_n$ be defined as follows: $x_1 \mathrel{\mathop:}= 1$, $x_2 \mathrel{\mathop:}= 2$, and $x_{n + 2} \mathrel{\mathop:}= \frac{1}{2}\left(x_{n+1} + x_n\right)$ for all $n \in \mathbb{N}$. Use the Principle of Strong Induction (1.2.5) to show that $1 \leq x_n \leq 2$ for all $n \in \mathbb{N}$.

\section*{Section 1.3 Exercises 2(c) and 4}
\subsection*{Exercise 2(c)}
Prove part (c) of Theorem 1.3.4.
\subsection*{Exercise 4}
Exhibit a bijection between $\mathbb{N}$ and the set of all odd integers greater than 13.
\subsubsection*{Solution}
The function $f : \mathbb{N} \rightarrow \left\{n : n > 13 \text{\quad and \quad $n$ is odd}\right\}$ where $f\left(n\right) = 2n + 13$ is a bijection. To demonstrate this, I will first show that it is \textbf{injective} and then show that it is \textbf{surjective}.

Let $n_1$ and $n_2$ be two numbers in $\mathbb{N}$ such that $f\left(n_1\right) = f\left(n_2\right)$. Using simple algebra, we can simplify.
\begin{align*}
2n_1 + 13 &= 2n_2 + 13 \\
2n_1 &= 2n_2 \\
n_1 &= n_2 \\
\end{align*}
It is clear that $f\left(n_1\right) = f\left(n_2\right)$ implies $n_1 = n_2$ and that $f$ is \textbf{injective}.

Let $y$ be any number in $\mathbb{N}$ such that $y = f\left(n\right) = 2n + 13$. We can express $n$ in terms of $y$ using trivial algebra.
\begin{align*}
n &= \frac{y - 13}{2} \\
\end{align*}
And we can substitute to check that our equation will allow us to find suitable $n$ to produce any $y \in \mathbb{N}$.
\begin{align*}
f\left(n\right) = y &= 2\left(\frac{y - 13}{2}\right) + 13 \\
&= y - 13 + 13 = y\\
\end{align*}
\end{flushleft}
\end{document}
