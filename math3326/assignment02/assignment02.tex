\documentclass[12pt]{article}
\special{papersize=8.5in,11in}
\usepackage[utf8]{inputenc}
\usepackage{mathtools}
\usepackage{amssymb,amsmath}
\pagestyle{plain}
\begin{document}
\begin{flushright}
{\Large
Jonathan Glines \\
MATH 3326 \\
Assignment 1 \\
}
\end{flushright}
\section*{Section 1.2 Exercises 7, 10, and 15}
\subsection*{Exercise 7}
Prove that $5^{2n} - 1$ is divisible by 8 for all $n \in \mathbb{N}$.
\subsubsection*{Solution}
We start with the statement $S \coloneqq $

\subsection*{Exercise 10}
Conjecture a formula for the sum $1/\left(1\cdot 3\right) + 1/\left(3 \cdot 5\right) + \cdots + 1 /\left(\left(2n - 1\right)\left(2n+1\right)\right)$, and prove your conjecture by using Mathematical Induction.
\subsubsection*{Solution}
After looking at some data, it appears that $\frac{1}{1 \cdot 3} + \frac{1}{3 \cdot 5} + \cdots + \frac{1}{\left(2n - 1\right)\left(2n + 1\right)} = \frac{n}{2n + 1}$.

First we note that this is true for $n = 1$, since $\left(1\right)/\left(2 \cdot 1 + 1\right) = 1/3$.

Now we assume that $\frac{1}{1 \cdot 3} + \frac{1}{3 \cdot 5} + \cdots + \frac{1}{\left(2n - 1\right)\left(2n + 1\right)} = \frac{n}{2n + 1}$ for a given $n \in \mathbb{N}$.
\begin{align*}
\frac{1}{1 \cdot 3} + \frac{1}{3 \cdot 5} + \cdots + \frac{1}{\left(2n - 1\right)\left(2n + 1\right)} &= \frac{n}{2n + 1} \\
\frac{1}{1 \cdot 3} + \frac{1}{3 \cdot 5} + \cdots + \frac{1}{\left(2n - 1\right)\left(2n + 1\right)} + \frac{1}{\left(2\left(n + 1\right) - 1\right)\left(2\left(n + 1\right) + 1\right)} &= \frac{n}{2n + 1} + \frac{1}{\left(2\left(n + 1\right) - 1\right)\left(2\left(n + 1\right) + 1\right)} \\
\end{align*}

\subsection*{Exercise 15}
Prove that $2n - 3 \leq 2^{n-2}$ for all $n \geq 5$, $n \in \mathbb{N}$.

\section*{Section 1.3 Exercises 2, 5, and 10}

\subsection*{Exercise 2}
Prove parts (b) and (c) of Theorem 1.3.4.
\begin{itemize}
\item[(b)] If $A$ is a set with $m \in \mathbb{N}$ elements and $C \subseteq A$ is a set with 1 element, then $A \setminus C$ is a set with $m - 1$ elements.
\subsubsection*{Solution}
Let $f$ be a bijection $f : \mathbb{N}_m \to A$ with $f(m) = x \in B$, which we know to exist because $A$ is finite.
Let $h$ be a function $h : \mathbb{N}_{m-1} \to A \setminus B$ such that $h(i) \coloneqq f(i)$ for $i \in \mathbb{N}_{m - 1}$.

For $h$ to be injective, we need to show that for all $x_1, x_2 \in \mathbb{N}_{m-1}$, if $h\left(x_1\right) = h\left(x_2\right)$, then $x_1 = x_2$. Because $h(x) = f(x)$ and $f$ is injective, we can see that $h\left(x_1\right) = f\left(x_1\right) = f\left(x_2\right) = h\left(x_2\right)$ and $h$ is indeed injective.

For $h$ to be surjective, we need to show that for any $b \in A \setminus B$ we can produce $x \in \mathbb{N}_{m-1}$ such that $h\left(x\right) = b$. Again leveraging the fact that $f$ is bijective, we know that since $A \setminus B \subseteq B$, $f$ must map onto every element in $A \setminus B$. This would mean that $h$ maps onto $A \setminus B$, except that we have to account for how we restriced in the domain of $h$. The only element in the domain of $f$ that we excluded from the domain of $A$ is $m$. Seeing that $h\left(m\right) \notin A \setminus B$ from the way we defined $f$, we conclude that the range of $h$ still contains $A \setminus B$ and that $h$ is surjective.

Having found $h$ to be a bijection from $\mathbb{N}_{m-1}$ onto $A \setminus B$, we conclude that $A \setminus B$ is finite.

\item[(C)] If $C$ is an infinite set and $B$ is a finite set, then $C \setminus B$ is an infinite set.
\subsubsection*{Solution}
Assume $C \setminus B$ is finite. Because $B \cap \left(C \setminus B\right) = \emptyset$, by Theorem 1.3.4 Part (a) we see that $B \cup \left(C \setminus B\right)$ has $n + m$ elements. It is apparent that $B \cup \left(C \setminus B\right) = B \cup C$, and that $B \cup C$ must also be fnitie. It is also clear that $C \subseteq B \cup C$. A contradiction arises when we apply Theorem 1.3.5 to show that $C$ is finite, which is contrary to the given fact that $C$ is an infinite set. Thus, by contradiction, we conclude that $C \setminus B$ is an infinite set, and that the implication originally given is true.

\end{itemize}

\subsection*{Exercise 5}
Give an explicit definition of the bijection $f$ from $\mathbb{N}$ onto $\mathbb{Z}$ described in Example 1.3.7(b).
\subsubsection*{Solution}
\begin{align*}
\mathbb{Z} &= \left\{0, 1, -1, 2, -2, 3, -3, \dots\right\} \\
f \coloneqq \left\{\stackrel{A}{B} \right.
\end{align*}
\subsection*{Exercise 10}
\begin{itemize}
\item[(a)] If $\left(m, n\right)$ is the 6th point down the 9th diagonal of the array in Figure 1.3.1, calculate its number according to the counting method given for Theorem 1.3.8.
\item[(b)] Given that $h\left(m, 3\right) = 19$, find $m$.
\end{itemize}
\end{document}
