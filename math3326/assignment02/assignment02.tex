\documentclass[12pt]{article}
\special{papersize=8.5in,11in}
\usepackage[utf8]{inputenc}
\usepackage{mathtools}
\usepackage{amssymb,amsmath}
\pagestyle{plain}
\begin{document}
\begin{flushright}
{\Large
Jonathan Glines \\
MATH 3326 \\
Assignment 1 \\
}
\end{flushright}
\section*{Section 1.2 Exercises 7, 10, and 15}
\subsection*{Exercise 7}
Prove that $5^{2n} - 1$ is divisible by 8 for all $n \in \mathbb{N}$.
\subsubsection*{Solution}
First we see that when $n = 1$, the expression $5^{2n} - 1 = 5^2 - 1 = 24 = 8(3)$ is divisible by 8.

Now given $n \in \mathbb{N}$, we assume that $5^{2n} - 1$ is divisible by 8. This is the same as saying that $5^{2n} - 1 = 8k$ for some $k \in \mathbb{Z}$. With some algebra, it can be shown that $5^{2\left(n + 1\right)} - 1$ is also divisible by 8.
\begin{align*}
5^{2n} - 1 &= 8k \\
5^{2n} \cdot 5^2 - 5^2 &= 8k \cdot 5^2 \\
5^{2\left(n + 1\right)} - 25 &= 8\left(5^2 k\right) \\
5^{2\left(n + 1\right)} - 1 &= 8\left(5^2 k\right) + 24 = 8\left(5^2 k + 3\right) \\
\end{align*}
And now we have $5^{2\left(n + 1\right)} - 1$ in the form $8h$ with $h = 5^2 k + 3 \in \mathbb{Z}$, thus showing that $5^{2\left(n + 1\right)} - 1$ is also divisible by 8.

By Mathematical Induction, $5^{2n} - 1$ must be divisible by 8 for all $n \in \mathbb{N}$.

\subsection*{Exercise 10}
Conjecture a formula for the sum $1/1\cdot 3 + 1/3 \cdot 5 + \cdots + 1 / \left(2n - 1\right)\left(2n+1\right)$, and prove your conjecture by using Mathematical Induction.

\subsection*{Exercise 15}
Prove that $2n - 3 \leq 2^{n-2}$ for all $n \geq 5$, $n \in \mathbb{N}$.
\subsubsection*{Solution}
First we check that $2n - 3 \leq 2^{n-2}$ for $n = 5$. This is true, since it works out to be $2(5) - 3 = 7 \leq 8 = 2^{5 - 2}$.

Now given $n \geq 5$, $n \in \mathbb{N}$, we assume that $2n - 3 \leq 2^{n-2}$ is true. With some algebra it can be shown that $2\left(n + 1\right) - 3 \leq 2^{\left(n + 1\right) - 2}$ must also be true.
\begin{align*}
2n - 3 &\leq 2^{n - 2} \\
2n - 3 + 2 &\leq 2^{n - 2} + 2 \\
2\left(n + 1\right) - 3 &\leq 2^{n - 1} = 2^{\left(n + 1\right) - 2} \\
\end{align*}

So we conclude by Mathematical Induction that $2n - 3 \leq 2^{n-2}$ for all $n \geq 5$, $n \in \mathbb{N}$.

\section*{Section 1.3 Exercises 2, 5, and 10}
\subsection*{Exercise 2}
Prove parts (b) and (c) of Theorem 1.3.4.
\subsection*{Exercise 5}
Give an explicit definition of the bijection $f$ from $\mathbb{N}$ onto $\mathbb{Z}$ described in Example 1.3.7(b).
\subsection*{Exercise 10}
\begin{itemize}
\item[(a)] If $\left(m, n\right)$ is the 6th point down the 9th diagonal of the array in Figure 1.3.1, calculate its number according to the counting method given for Theorem 1.3.8.
\item[(b)] Given that $h\left(m, 3\right) = 19$, find $m$.
\end{itemize}
\end{document}
