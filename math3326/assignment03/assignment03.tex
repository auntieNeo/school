\documentclass[12pt]{article}
\special{papersize=8.5in,11in}
\usepackage[utf8]{inputenc}
\usepackage{mathtools}
\usepackage{amssymb,amsmath}
\pagestyle{plain}
\begin{document}
\begin{flushright}
{\Large
Jonathan Glines \\
MATH 3326 \\
Assignment 3 \\
}
\end{flushright}
\section*{Section 2.1 Exercises 4, 8, 9, and 16}
\subsection*{Exercise 4}
If $a \in \mathbb{R}$ satisfies $a \cdot a = a$, prove that either $a = 0$ or $a = 1$.

\subsection*{Exercise 8}
\begin{itemize}
\item[(a)] Show that if $x, y$ are rational numbers, then $x + y$ and $xy$ are rational numbers.
\item[(b)] Prove that if $x$ is a rational number and $y$ is an irrational number, then $x + y$ is an irrational number. If, in addition, $x \neq 0$, then show that $xy$ is an irrational number.
\end{itemize}

\subsection*{Exercise 9}
Let $K \coloneqq \left\{s + t\sqrt{2} : s, t \in \mathbb{Q}\right\}$. Show that $K$ satisfies the following:
\begin{itemize}
\item[(a)] If $x_1, x_2 \in K$, then $x_1 + x_2 \in K$ and $x_1x_2 \in K$.
\item[(b)] If $x \neq 0$ and $x \in K$, then $1/x \in K$.
\end{itemize}
(Thus, the set $K$ is a {\it subfield} of $\mathbb{R}$. With the order inherited from $\mathbb{R}$, the set $K$ is an ordered field that lies between $\mathbb{Q}$ and $\mathbb{R}$.

\subsection*{Exercise 16}
Find all real numbers $x$ that satisfy the following inequalities.
\begin{itemize}
\item[(a)] $x^2 > 3x + 4$,
\item[(b)] $1 < x^2 < 4$,
\item[(c)] $1/x < x$,
\item[(d)] $1/x < x^2$.
\end{itemize}

\section*{Section 2.2 Exercises 5, 9, and 17}
\subsection*{Exercise 5}
If $a < x < b$ and $a < y < b$, show that $\left|x - y\right| < b - a$. Interpret this geometrically.

\subsection*{Exercise 9}
Find all values of $x$ that satisfy the follomwing inequalities. Sketch graphs.
\begin{itemize}
\item[(a)] $\left| x - 2\right| \leq x + 1$,
\item[(b)] $3\left|x\right| \leq 2 - x$.
\end{itemize}

\subsection*{Exercise 17}
Show that if $a, b \in \mathbb{R}$, and $a \neq b$, then there exist $\varepsilon$-neighborhoods $U$ of $a$ and $V$ of $b$ such that $U \cap V = \emptyset$.
\end{document}
