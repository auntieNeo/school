\documentclass[12pt]{article}
\special{papersize=8.5in,11in}
\usepackage[utf8]{inputenc}
\usepackage{mathtools}
\usepackage{amssymb,amsmath,amsthm}
\newtheorem*{thm}{Theorem}
\newtheorem{lem}{Lemma}
\pagestyle{plain}
\begin{document}
\begin{flushright}
{\Large
Jonathan Glines \\
MATH 3326 \\
Assignment 3 \\
}
\end{flushright}
\section*{Section 2.1 Exercises 4, 8, 9, and 16}
\subsection*{Exercise 4}
If $a \in \mathbb{R}$ satisfies $a \cdot a = a$, prove that either $a = 0$ or $a = 1$.
\subsubsection*{Solution}
Since this section focuses on proving things using the field axioms, first I'm going to introduce a small lemma using only these field axioms.
\begin{lem}
For any $a, b, c \in \mathbb{R}$, $a = b \implies a \cdot c = b \cdot c$.
\end{lem}
\begin{proof}
This is trivial. Knowing that that $a \cdot c = a \cdot c$ and that $a = b$, we replace one of the $a$'s with $b$ and conclude that $a \cdot c = b \cdot c$.
\end{proof}

\begin{thm}
If $a \in \mathbb{R}$ satisfies $a \cdot a = a$, then $a = 0$ or $a = 1$.
\end{thm}
\begin{proof}
Consider any $a \in \mathbb{R}$. For $a \neq 0$, the following can be shown by the field axioms:
\begin{align*}
a \cdot a &= a \\
\left(a \cdot a\right) \cdot \left(1/a\right) &= a \cdot \left(1 / a\right) & \text{\bf Lemma 1}\\
a \cdot \left(a \cdot \left(1/a\right)\right) &= a \cdot \left(1/a\right) & \text{\bf (M2)} \\
a \cdot 1 &= 1 & \text{\bf (M4)}\\
a &= 1 & \text{\bf (M3)}
\end{align*}
Thus, whenever $a \neq 0$, we must have $a = 1$.

Now for the case where $a = 0$, where the multiplicative inverse $\left(1/a\right)$ does not exist, we see that the equation still holds:
\begin{align*}
a \cdot a &= a \\
a \cdot 0 &= a \\
0 &= a & \text{\bf Theorem 2.1.2 (c)}
\end{align*}
Having considered every possibility for $a$, we conclude that either $a = 0$ or $a = 1$.
\end{proof}

\subsection*{Exercise 8}
\begin{itemize}
\item[(a)] Show that if $x, y$ are rational numbers, then $x + y$ and $xy$ are rational numbers.
\item[(b)] Prove that if $x$ is a rational number and $y$ is an irrational number, then $x + y$ is an irrational number. If, in addition, $x \neq 0$, then show that $xy$ is an irrational number.
\end{itemize}
\subsubsection*{Solution}
\begin{lem}
The set $\mathbb{Z} \setminus \left\{0\right\}$ is closed under multiplication.
\end{lem}
\begin{proof}
Let $x, y$ be elements in $\mathbb{Z} \setminus \left\{0\right\}$. Because $\mathbb{Z} \setminus \left\{0\right\}$ is a subset of $\mathbb{Z}$ and $\mathbb{Z}$ is closed under multiplication, the product $xy \in \mathbb{Z}$. The only element in $\mathbb{Z}$ that is not in $\mathbb{Z} \setminus \left\{0\right\}$ is 0, so in order to show that $xy \in \mathbb{Z} \setminus \left\{0\right\}$ it is sufficient to show that $xy \neq 0$.

Suppose to the contrary that $xy = 0$. Let $x_R, y_R$ be elements in $\mathbb{R}$ such that $x_R = x$ and $y_R = y$. We substitute $x_R$ and $y_R$ for $x$ and $y$ respectively so we can use the reciprocal property of $\mathbb{R}$.
\begin{align*}
xy = \left(x_R\right) \left(y_R\right) &= 0 \\
\left(x_R\right) \left(y_R\right) \left(y_R\right)^{-1} &= 0\cdot\left(y_R\right)^{-1} \\
x_R &= 0 = x
\end{align*}
This contradicts our original assumption that $x \in \mathbb{Z} \setminus \left\{0\right\}$. Thus, $xy \neq 0$, and $xy \in \mathbb{Z} \setminus \left\{0\right\}$.
\end{proof}
\begin{itemize}
\item[(a)]
\begin{thm}
If $x, y$ are rational numbers, then $x + y$ and $xy$ are rational numbers.
\end{thm}
\begin{proof}
Given $x, y \in \mathbb{Q}$, we know from the definition of $\mathbb{Q}$ that for some $p,q,r,s \in \mathbb{Z}$ with $q \neq 0$ and $s \neq 0$, we have $x = \frac{p}{q}$ and $y = \frac{r}{s}$. By substituting for $x$ and $y$ we see that $x + y = \frac{p}{q} + \frac{r}{s} = \frac{ps + rq}{qs}$ and that $xy = \frac{p}{q} \cdot \frac{r}{s} = \frac{pr}{qs}$. Because $\mathbb{Z}$ is closed under both addition and multiplication, both $ps + rq$ and $pr$ must reside in $\mathbb{Z}$. The difficulty in proving that $x + y$ and $xy$ are in $\mathbb{Q}$ is in showing that $qs \neq 0$, i.e. $qs \in \mathbb{Z} \setminus \left\{0\right\}$. Since $q, s \in \mathbb{Z} \setminus \left\{0\right\}$ and $\mathbb{Z} \setminus \left\{0\right\}$ is closed under multiplication from Lemma 2, $qs$ must also be in $\mathbb{Z} \setminus \left\{0\right\}$. Having found both $x + y$ and $xy$ in the form of the elements of $\mathbb{Q}$, we conclude that both $x + y$ and $xy$ are rational numbers.
\end{proof}
\item[(b)]
\begin{thm}
If $x$ is a rational number and $y$ is an irrational number, then $x + y$ is an irrational number. If, in addition, $x \neq 0$, then $xy$ is an irrational number.
\end{thm}
\begin{proof}
Given $x \in \mathbb{Q}$ and $y \notin \mathbb{Q}$, suppose that $x + y \in \mathbb{Q}$. We can write $x + y = \frac{p}{q}$ for some $p, q \in \mathbb{Z}$ with $q \neq 0$. From the definition of $\mathbb{Q}$, we know that $x = \frac{r}{s}$ for some $r, s \in \mathbb{Z}$ with $s \neq 0$. We solve for $y$ to get $y$ in the form of a rational number:
\begin{align*}
x + y &= \frac{p}{q} \\
y &= \frac{p}{q} - \frac{r}{s} = \frac{ps - rq}{qs} \\
\end{align*}
Since we showed that the rational numbers are closed under addition in part (a), this suggests that $y$ is a rational number. Since this contradicts the given premise that $y$ is irrational, we conclude that $x + y$ must be irrational.

Given $x \in \mathbb{Q}$ and $y \notin \mathbb{Q}$ with $x \neq 0$, suppose that $xy \in \mathbb{Q}$. We can write $xy = \frac{p}{q}$ for some $p, q \in \mathbb{Z}$ with $q \neq 0$. From the definition of $\mathbb{Q}$, we know that $x = \frac{r}{s}$ for some $r, s \in \mathbb{Z}$ with $s \neq 0$. Knowing that $x \neq 0$, we can use its inverse when we solve for $y$ and get $y$ in the form of a rational number:
\begin{align*}
xy &= \frac{p}{q} \\
\frac{r}{s} \cdot y &= \frac{p}{q} \\
y &= \frac{p}{q} \cdot \frac{s}{r} = \frac{ps}{qr} \\
\end{align*}
Again drawing from the work in part (a), we know that the rational numbers are closed under multiplication, which would suggest here that $y$ is a rational number. Since this contradicts the given premise that $y$ is irrational, we conclude that $xy$ must be irrational.
\end{proof}
\end{itemize}

\subsection*{Exercise 9}
Let $K \coloneqq \left\{s + t\sqrt{2} : s, t \in \mathbb{Q}\right\}$. Show that $K$ satisfies the following:
\begin{itemize}
\item[(a)] If $x_1, x_2 \in K$, then $x_1 + x_2 \in K$ and $x_1x_2 \in K$.
\item[(b)] If $x \neq 0$ and $x \in K$, then $1/x \in K$.
\end{itemize}
(Thus, the set $K$ is a {\it subfield} of $\mathbb{R}$. With the order inherited from $\mathbb{R}$, the set $K$ is an ordered field that lies between $\mathbb{Q}$ and $\mathbb{R}$.)
\subsubsection*{Solution}
\begin{itemize}
\item[(a)] Let $x_1, x_2 \in K$. From the definition of $K$, for some $s_1, t_1, s_2, t_2 \in \mathbb{Q}$ we have $x_1 = s_1 + t_1 \sqrt{2}$ and $x_2 = s_2 + t_2\sqrt{2}$. With some algebra we get $x_1 + x_2 = s_1 + s_2 + \left(t_1 + t_2\right)\sqrt{2}$. Since this is of the form $s + t\sqrt{2}$ with $s_1 + s_2 = s \in \mathbb{Q}$ and $t_1 + t_2 = t \in \mathbb{Q}$, we conclude that $x_1 + x_2 \in K$.

Similarly, with even more algebra we get
\begin{align*}
x_1x_2 &= \left(s_1 + t_1\sqrt{2}\right) \left(s_2 + t_2\sqrt{2}\right) \\
&= s_1s_2 + t_1s_2\sqrt{2} + s_1t_2\sqrt{2} + 2t_1t_2 \\
&= \left(s_1s_2 + 2t_1t_2\right) + \left(t_1s_2 + s_1t_2\right)\sqrt{2} \\
\end{align*}
which is of the form $s + t \sqrt{2}$ with $s_1s_2 + 2t_1t_2 = s \in \mathbb{Q}$ and $t_1s_2 + s_1t_2 = t \in \mathbb{Q}$, so we conclude that $x_1x_2 \in K$.
\item[(b)] Let $x \in K$ with $x \neq 0$. From the definition of $K$, for some $s, t \in \mathbb{Q}$ we have $x = s + t\sqrt{2}$.
\begin{align*}
x &= s + t\sqrt{2} \\
\frac{1}{x} &= \frac{1}{s + t\sqrt{2}} = \\ 
\end{align*}
\end{itemize}

\subsection*{Exercise 16}
Find all real numbers $x$ that satisfy the following inequalities.
\begin{itemize}
\item[(a)] $x^2 > 3x + 4$,

Find the points where $x^2$ and $3x + 4$ intersect:
\begin{align*}
x^2 &= 3x + 4 \\
0 &= -x^2 + 3x + 4 \\
&= \left(-x + 4\right) \left(x + 1\right) & \text{$x = 4$ or $x = -1$}\\
\end{align*}
Consider each interval:

For $\left(-\infty, -1\right)$, we have $x^2 > 3x+4$.

For $\left(-1, 4\right)$, we have $x^2 < 3x+4$.

For $\left(4, \infty\right)$, we have $x^2 > 3x+4$.

The real numbers $x$ that satisfy $x^2 > 3x + 4$ are $x \in \left(-\infty, -1\right) \cup \left(4, \infty\right)$.
\item[(b)] $1 < x^2 < 4$,

$x$ intersects the horizontal lines 1 and 4 when $x = \pm 1$ and $x = \pm 2$.

Consider each of the five intervals:

For $\left(-\infty, -2\right)$, the inequality is not satisfied because $x^2 > 4$.

For $\left(-2, -1\right)$, the inequality is satisfied.

For $\left(-1, 1\right)$, the inequality is not satisfied because $x^2 < 1$.

For $\left(1, 2\right)$, the inequality is satisfied.

For $\left(2, \infty\right)$, the inequality is not satisfied because $x^2 > 4$.

The real numbers $x$ that satisfy the inequality $1 < x^2 < 4$ are $x \in \left(-2, -1\right) \cup \left(1, 2\right)$.
\item[(c)] $1/x < x$,

Note that $x = 0$ is a point of interest, since $1/x$ is undefined there.

Find the points where $1/x$ and $x$ intersect:
\begin{align*}
1/x &= x \\
0 &= x - 1/x & \text{$x = 1$ or $x = -1$}
\end{align*}

Consider each of the intervals:

For $\left(-\infty, -1\right)$, the inequality is not satisfied.

For $\left(-1, 0\right)$, the inequality is satisfied.

For $\left(0, 1\right)$, the inequality is not satisfied.

For $\left(1, \infty\right)$, the inequality is satisfied.

The real numbers $x$ that satisfy the inequality $1/x < x$ are $x \in \left(-1, 0\right) \cup \left(1, \infty\right)$.
\item[(d)] $1/x < x^2$.

Note that $x = 0$ is a point of interest, since $1/x$ is undefined there.

Find the point where $1/x$ and $x^2$ intersect:
\begin{align*}
1/x &= x^2 \\
1 &= x^3 \\
0 &= x^3 - 1 & x = 1\\
\end{align*}

Consider each of the intervals:

For $\left(-\infty, 0\right)$, the inequality is satisfied.

For $\left(0, 1\right)$, the inequality is not satisfied.

For $\left(1, \infty\right)$, the inequality is satisfied.

The real numbers $x$ that satisfy the inequality $1/x < x^2$ are $x \in \left(-\infty, 0\right) \cup \left(1, \infty\right)$.

\end{itemize}

%\section*{Section 2.2 Exercises 5, 9, and 17}
%\subsection*{Exercise 5}
%If $a < x < b$ and $a < y < b$, show that $\left|x - y\right| < b - a$. Interpret this geometrically.
%
%\subsection*{Exercise 9}
%Find all values of $x$ that satisfy the follomwing inequalities. Sketch graphs.
%\begin{itemize}
%\item[(a)] $\left| x - 2\right| \leq x + 1$,
%\item[(b)] $3\left|x\right| \leq 2 - x$.
%\end{itemize}
%\subsubsection*{Solution}
%\begin{itemize}
%\item[(a)] I found them: $S \coloneqq \left\{x \in \mathbb{R} : \left|x - 2\right| \leq x + 1\right\}$
%\end{itemize}
%
%\subsection*{Exercise 17}
%Show that if $a, b \in \mathbb{R}$, and $a \neq b$, then there exist $\varepsilon$-neighborhoods $U$ of $a$ and $V$ of $b$ such that $U \cap V = \emptyset$.

\end{document}
