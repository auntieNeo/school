\documentclass[12pt]{article}
\special{papersize=8.5in,11in}
\usepackage[utf8]{inputenc}
\usepackage{mathtools}
\usepackage{amssymb,amsmath,amsthm}
\newtheorem*{thm}{Theorem}
\newtheorem{lem}{Lemma}
\pagestyle{plain}
\begin{document}
\begin{flushright}
{\Large
Jonathan Glines \\
MATH 3326 \\
Assignment 4 \\
}
\end{flushright}
\section*{Section 2.2 Exercises 5, 9, and 17}
\subsection*{Exercise 5}
If $a < x < b$ and $a < y < b$, show that $\left|x - y\right| < b - a$. Interpret this geometrically.

\subsubsection*{Solution}
$\left|x - y\right| \in \mathbb{P}$ is positive.

\subsection*{Exercise 9}
Find all the values of $x$ that satisfy the following inequalities. Sketch graphs.
\begin{itemize}
\item[(a)] $\left|x - 2\right| \leq x + 1$,
\item[(b)] $3\left|x\right| \leq 2 - x$.
\end{itemize}

\subsection*{Exercise 17}
Show that if $a, b \in \mathbb{R}$, and $a \neq b$, then there exist $\varepsilon$-neighborhoods $U$ of $a$ and $V$ of $b$ such that $U \cap V = \emptyset$.

\section*{Section 2.3 Exercises 5 and 9}
\subsection*{Exercise 5}
Find the infimum and supremum, if they exist, of each of the following sets.
\begin{itemize}
\item[(a)] $A \coloneqq \left\{x \in \mathbb{R} : 2x + 5 > 0\right\}$,
\item[(b)] $B \coloneqq \left\{x \in \mathbb{R} : x + 2 \geq x^2\right\}$,
\item[(c)] $C \coloneqq \left\{x \in \mathbb{R} : x < 1/x\right\}$,
\item[(d)] $D \coloneqq \left\{x \in \mathbb{R} : x^2 - 2x - 5 < 0\right\}$.
\end{itemize}

\subsection*{Exercise 9}
Let $S \subseteq \mathbb{R}$ be nonempty. Show that if $u = \sup S$, then for every number $n \in \mathbb{N}$ the number $u - 1/n$ is not an upper bound of $S$, but the number $u + 1/n$ is an upper bound of $S$. (The converse is also true; see Exercise 2.4.3.)

\section*{Section 2.4 Exercise 3}
\subsection*{Exercise 3}
Let $S \subseteq \mathbb{R}$ be nonempty. Prove that if a number $u$ in $\mathbb{R}$ has the properties: (i) for every $n \in \mathbb{N}$ the number $u - 1/n$ is not an upper bound of $S$, and (ii) for every number $n \in \mathbb{N}$ the number $u + 1/n$ is an upper bound of $S$, then $u = \sup S$. (This is the converse of Exercise 2.3.9.)
\end{document}
