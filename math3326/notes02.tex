\documentclass[12pt]{article}
\special{papersize=8.5in,11in}
\usepackage[utf8]{inputenc}
\usepackage{amssymb,amsmath,amsthm}
\newtheorem*{definition}{Definition}
\pagestyle{plain}
\begin{document}
\begin{flushright}
\Large{Jonathan Glines}
\end{flushright}
\section*{Notes}
\underline{Sequences and their limits}

Sequences in $\mathbb{R}$: an infinite list $x_1, x_2, x_3, \cdots$ of real numbers.

$1, 1 \frac{1}{2}, 1 \frac{3}{4}, 1 \frac{7}{8}, \cdots \quad x_n = 2 - \frac{1}{2^{n-1}}$.

$1, -1, 1, -1, \cdots \quad x_n = \left(-1\right)^{n + 1}$

$1, 1, 2, 3, 5, 8, 13, \cdots x_1 = 1, x_2 = 1, x_{n + 1} = x_n + x_{n + 1} (n \geq 2)$ 

$\frac{1}{2} , \frac{1 \cdot 3}{2 \cdot 4}, \frac{1 \cdot 3\cdot 5}{2\cdot 4 \cdot 6}, \cdots$...

Really, a \underline{function} $n \rightarrow x_n \left(n \in \mathbb{N}\right)$.

Many notations:
\begin{itemize}
\item $x = (x_n)$
\item $\{x_n\}_{n=1}^\infty$
\item $x_1, x_2, x_3, \cdots$
\end{itemize}

Indexing \underline{can} start with something other than 1.

$0, -1, 2, -3, 4, \cdots$

$x_n=(-1)^n n, n \geq 0$

\underline{Limit of a sequence}

\underline{example}
\begin{itemize}
\item[a] $1, 1\frac{1}{2}, 1 \frac{3}{4}, 1\frac{7}{8}, \cdots$

$\lim_{n \to \infty} x_n = 2$

$x_n \rightarrow 2$

Sequences approaches 2, converges to 2, limit is 2

\item[b] $0, 1, 0, \frac{1}{2}, 0, \frac{1}{3}, \cdots$
Converges to 0.

\item[c] $1, 3, 5, 7, \cdots$  Diverges.

\item[d] $1, -1, 1, -1, \cdots$ Diverges.
\end{itemize}

Need a definition of a sequence $x_1, x_2, \cdots$ converging to a real number $x$.

\underline{Far out} in the sequence, all the terms are \underline{close to} $x$.

For any interval $(x - \epsilon, x + \epsilon)$, $\epsilon > 0$, the sequence eventually gets into the interval and stays there.

The book calls the epsilon interval a neighborhood, $\left(x - \epsilon, x + \epsilon \right) = V_\epsilon\left(x\right)$.

\begin{definition}
A sequence $x_1, x_2, \cdots$ converges to the real number $x$ if, for every $\epsilon > 0$, there exists $K \in \mathbb{N}$ such that $n \geq K \implies \left|x_n - x\right| < \epsilon$.
\end{definition}

\underline{Notes}
\begin{enumerate}
\item A sequence can converge to at most one number $x$. (Proved in a minute) Say $\lim_{n \to \infty} x_n = x$. (or $x_n \rightarrow x$)

\item $\lim_{n \to \infty} x_n$ exists same as saying sequence converges.

\item Sequences can diverge in many ways.

$1, 3, 5, \cdots$ diverges to $\infty$

$1, -1, 1, -1, 1, -1, \cdots$

$0, 1, 0, 0, 1, 0, 0, 0, 1, \cdots$

$\sin n, n \geq 1$
\end{enumerate}

\underline{Proof of uniqueness} Suppose that $x_n \to x$ and $x_n \rightarrow y$. Must prove that $x = y$.

Suppose that $x \neq y$.

Let $\epsilon = \frac{\left|y - x\right|}{2}$. Because $x_n \to x$, there exists $K$, such that $n \geq K_1 \implies \left|x_n - x\right| < \epsilon$. Because $x_n \to y$, there exists $K_2 \in \mathbb{N}$ such that $n \geq K_2 \implies \left|x_n - y\right| < \epsilon$.

Let $n = \max\left\{K_1, K_2\right\}$. Then
\begin{align*}
\left|y - x\right| &= \left|\left(y - x_n\right) - \left(x - x_n\right)\right| \\
&\leq \left|y - x_n \right| + \left|x - x_n\right| \\
&< \epsilon + \epsilon\\
&= 2\epsilon = \left|y - x\right|
\end{align*}

\underline{Proving particular limit statements}

\underline{Examples} Prove the statement.
\begin{itemize}
\item[(a)] $\lim_{n \to \infty}\frac{n}{n+1} = 1$.

\underline{Scratch work}: Given $\epsilon > 0$, must produce $K$ so that $n \geq K \implies \left|\frac{n}{n + 1} - 1\right| < \epsilon$

Do the algebra $\left|\frac{n}{n + 1} - 1\right| = \left|\frac{-1}{n+1}\right| = \frac{1}{n+1} < \epsilon$.

\begin{proof}
Given $\epsilon > 0$, let $K \in \mathbb{N}$ be such that $K > \frac{1}{\epsilon} - 1$.

If $n \geq K$, then
\begin{align*}
n \geq K &> \frac{1}{\epsilon} - 1\\
n + 1 &> \frac{1}{\epsilon}\\
\frac{1}{n + 1} &< \epsilon \\
\left|\frac{n}{n+1} - 1\right| < \epsilon
\end{align*}
\end{proof}
\end{itemize}
\end{document}
