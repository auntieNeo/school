\documentclass[12pt]{article}
\special{papersize=8.5in,11in}
\usepackage[utf8]{inputenc}
\usepackage{amssymb,amsmath}
\pagestyle{plain}
\begin{document}
\begin{flushright}
\Large{Jonathan Glines}
\end{flushright}
\begin{flushleft}
\section*{Notes}
Monday Completeness of $\mathbb{R}$.

Terms: bounded above, bounded below, bounded upper and lower bounds.

Completeness. Every nonempty set $S \subseteq \mathbb{R}$ that is bounded above has a least upper bound.

Unique; called $\text{sup}S$

Greatest lower bound, too (``$\text{inf}S$")

Examples

Proofs involving sup or inf.

\subsection*{More on Completeness}
\subsubsection*{Sample problem (#10 Section 2.3)}
Suppose that $A, B \subseteq \mathbb{R}$ are nonempty and bounded above. Show that $A \cup B$ is bounded above and that its supremum equals $\text{max}\{\text{sup}A, \text{sup}B\}$.

\textbf{Proof} Let $\alpha = \text{sup}A$ and $\beta = \text{sup}B$. To be definite, may suppose that $\beta \geq \alpha$.

Want to show that $\text{sup}\left(A \cup B\right) = \beta$.

Must show exactly two things.

\begin{enumerate}
\item $\beta$ is an upper bound for $A \cup B$.
\item If $u$ any upper bound for $A \cup B$, then $\beta \leq u$.
\end{enumerate}

1. Every element $x \in A \cup B$ is either in $B$ or $A$.
$x \in B \implies x \leq \beta \text{Because $\beta$ is an upper bound for $B$.}$

$x \in A \implies x \leq \alpha \leq \beta \text{Because $\alpha$ is an upper bound for $A$.}$

2. Suppose that $u$ is an upper bound for $A \cup b$. Then $u$ is an upper bound for $B$. ($b \in B \implies b \in A\cupB \implies b \leq u$.) Because $\beta$ is the least upper bound for $B$. Follows that $\beta \leq u$.

\subsection*{Problem}
$A, B \subseteq \mathbb{R}$ nonempty.
Suppose that $a \leq b$ for all $a \in A, b\in B$.

Prove that $A$ is bounded above, $B$ bounded below, and that $\text{sup}A \leq \text{inf}B$.
\subsubsection*{Proof}
$A$ bounded above (by any element of $B$). $B$ bounded below (by any element of $A$).

Let $\alpha = \text{sup}A$ and $\beta = \text{inf}B$. Want to show that $\alpha \leq \beta$.

Strategy. Show that $\beta$ is an upper bound for $A$.

How? Show that $a \leq \beta$ for all $a \in A$.

Consider an element $a \in A$. Because $a \leq b$ for all $b\in B$, $a$ is a lower bound for $B$. Because $\beta$ is the \textbf{greatest} lower bound, $a \leq \beta$.

Because this holds for all $a \in A$, $\beta$ is an upper bound for $A$.

Because $\alpha$ is the least upper bound for $A$, $\alpha \leq \beta$.

\subsection*{``Applications"}
A real-value function $f$ is said to be \underline{bounded above}. In other words, there is a number $u$ such that $f(x) \leq u$ for all $x \in U$.

Similarly for ``bounded below", ``bounded".

\subsubsection*{Examples}
\begin{itemize}
\item[a.] $f(x) = e^x$

Not bounded above
Bounded below by $0$, for example.

$\text{inf}f = 0$

\item[b.] $$f(x) = \frac{1}{x^2 + 1}$$

Bounded. Bounded above by 1, bounded below by $-6$, for example.

$\text{sup}f = 1$
$\text{inf}f = 0$
\end{itemize}

\underline{Example} $f, g \mathbb{R} \rightarrow \mathbb{R}$ both bounded abov. Prove that $text{sup}(f + g) \leq \text{sup}f + \text{sup} g$, and give an example where $<$ holds.

Set up: Let $\alpha = \text{sup}f$ and $\beta = \text{sup}g$.

Want to show that $\text{sup}(f + g) \leq \alpha + \beta$.

Enough to show that $\alpha + \beta$ is an upper bound for $f + g$.

$\alpha$ an upper bound for $f$: $f(x) \leq \alpha$ for all $x \in \mathbb{R}$

\end{flushleft}
\end{document}
