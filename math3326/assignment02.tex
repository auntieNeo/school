\documentclass[12pt]{article}
\special{papersize=8.5in,11in}
\usepackage[utf8]{inputenc}
\usepackage{amssymb,amsmath,amsthm}
\newtheorem*{thm}{Theorem}
\newtheorem{lem}{Lemma}
\usepackage{colonequals}
\usepackage{parskip}
\pagestyle{plain}
\begin{document}
\begin{flushright}
\Large{Jonathan Glines}
\end{flushright}
\section*{Section 2.1 Exercises 5, 8, 15(b), and 18}
\subsection*{Exercise 5}
If $a \ne 0$ and $b \ne 0$, show that $1/\left(ab\right) = \left(1/a\right)\left(1/b\right)$.

\subsubsection*{Notes}
Very abstract. Show $\left(ab\right)^{-1} = a^{-1}b^{-1}$. Know \underline{only} that: $\left(ab\right)^{-1}\cdot\left(ab\right) = 1$ ($(ab)(ab)^{-1} = 1$ also)
\[
a^{-1} a = 1,\quad
b^{-1} b = 1
\]
Suggestion:
\begin{enumerate}
\item Show that $a^{-1}b^{-1}$ has the same property that $(ab)^{-1}$ does, namely $(a^{-1}b^{-1})(ab) = 1$.
\item Somehow conclude that $(ab)^{-1} = a^{-1}b^{-1}$.
\end{enumerate}

\subsubsection*{Solution}
\begin{lem}
$\left(ab\right)\left(cd\right) = \left(ac\right)\left(bd\right)$ for all $a, b, c, d \in \mathbb{R}$.
\end{lem}

\begin{proof}
TODO.
\end{proof}

\begin{thm}
$\left(ab\right)^{-1} = a^{-1}b^{-1}$ for all $a, b \in \mathbb{R}$.
\end{thm}

\begin{proof}
First, notice that because $\left(ab\right)\left(ab\right)^{-1} = 1$, we can apply property (M3) in the following way.
\begin{align*}
\left(a^{-1}b^{-1}\right) &= \left(a^{-1}b^{-1}\right)\left(\left(ab\right)\left(ab\right)^{-1}\right) & \text{(M3)}\\
\intertext{Now we rearange this using properties (M2), }
&= \left(\left(a^{-1}b^{-1}\right)\left(ab\right)\right)\left(ab\right)^{-1} & \text{(M2)}
\intertext{From here, our strategy will be to rearrange the factor on the left by un-associating its factors using properties (M3) and (M2), and then moving those factors using property (M1).}
&= \left(\left[\left(a^{-1}b^{-1}\right) \cdot 1\right]\left(ab\right)\right)\left(ab\right)^{-1} & \text{(M3)}\\
&= \left(\left[\left(a^{-1}\right)\left(b^{-1}\cdot1\right)\right]\left(ab\right)\right)\left(ab\right)^{-1} & \text{(M2)}\\
&= \left(\left[\left(a^{-1}\right)\left(b^{-1}\right)\right]\left(ab\right)\right)\left(ab\right)^{-1} & \text{(M3)}\\
&= \left(\left[\left(a^{-1}\right)\left(b^{-1}\right)\right]\left[\left(ab\right) \cdot 1\right]\right)\left(ab\right)^{-1} & \text{(M3)}\\
&= \left(\left[\left(a^{-1}\right)\left(b^{-1}\right)\right]\left[\left(a\right) \left(b \cdot 1\right)\right]\right)\left(ab\right)^{-1} & \text{(M2)}\\
&= \left(\left[\left(a^{-1}\right)\left(b^{-1}\right)\right]\left[\left(a\right) \left(b\right)\right]\right)\left(ab\right)^{-1} & \text{(M3)}\\
\end{align*}
\end{proof}

\subsection*{Exercise 8}
\begin{itemize}
\item[(a)] Show that if $x, y$ are rational numbers, then $x + y$ and $xy$ are rational numbers.
\item[(b)] Prove that if $x$ is a rational number and $y$ is an irrational number, then $x + y$ is an irrational number. If, in addition, $x \neq 0$, then show that $xy$ is an irrational number.
\end{itemize}

\subsubsection*{Notes}
\begin{itemize}
\item[(a)] $x, y \in \mathbb{Q} \implies x + y \in \mathbb{Q} \text{ and } xy \in \mathbb{Q}$
\item[(b)] $x \in \mathbb{Q}, y \notin \mathbb{Q} \implies x + y \notin \mathbb{Q}$
\item[(c)] $x \in \mathbb{Q} \setminus \{0\}, y \notin \mathbb{Q} \implies xy \notin \mathbb{Q}$
\end{itemize}
Items (b) and (c) can be proved by contradiction.\\
\underline{Prove by contradiction}\\
Given that $x \in \mathbb{Q}$ and $y \notin \mathbb{Q}$, assume that $x + y \in \mathbb{Q}$. Find a contradiction.

\subsubsection*{Solution}
\begin{itemize}
\item[(a)]
\begin{lem}
The set $\mathbb{Z} \setminus \left\{0\right\}$ is closed under multiplication.
\end{lem}

\begin{proof}
TODO.
\end{proof}

\begin{thm}
If $x, y$ are rational numbers, then $x + y$ and $xy$ are rational numbers.
\end{thm}

\begin{proof}
Let $x, y$ be rational numbers. From the definition of the set of rational numbers, $x = x_1/x_2$ and $y = y_1/y_2$ for some $x_1, y_1 \in \mathbb{Z}$ and some $x_2, y_2 \in \mathbb{Z} \setminus \left\{0\right\}$.

The number $x + y$ would then equal $\left(x_1y_2 + y_1x_2\right)/\left(x_2 \cdot y_2\right)$. Because $x_1, y_1, x_2, y_2 \in \mathbb{Z}$ and $\mathbb{Z}$ is closed under both addition and multiplication, $x_1y_2 + y_1x_2 \in \mathbb{Z}$. 
Because $x_2, y_2 \in \mathbb{Z}\setminus\left\{0\right\}$ and $\mathbb{Z}\setminus\left\{0\right\}$ is closed under multiplication (Lemma 2), $x_2 \cdot y_2 \in \mathbb{Z} \setminus \left\{0\right\}$. Knowing that $x_1y_2 + y_1x_2 \in \mathbb{Z}$ and that $x_1 \cdot y_1 \in \mathbb{Z} \setminus \left\{0\right\}$, we see that $x + y$ is of the form $u / v$ where $u \in \mathbb{Z}$ and $v \in \mathbb{Z} \setminus \left\{0\right\}$. Therefore, $x + y \in \mathbb{Q}$.

The number $xy$ equals $\left(x_1 \cdot y_1\right) / \left(x_2 \cdot y_2\right)$. Because $x_1, y_1 \in \mathbb{Z}$ and $\mathbb{Z}$ is closed under multiplication, $x_1 \cdot y_1 \in \mathbb{Z}$. Likewise, because $x_2, y_2 \in \mathbb{Z} \setminus \left\{0\right\}$ and $\mathbb{Z} \setminus \left\{0\right\}$ is closed under multiplication (Lemma 2), $x_2 \cdot y_2 \in \mathbb{Z} \setminus \left\{0\right\}$. Knowing that $x_1 \cdot y_1 \in \mathbb{Z}$ and that $x_2 \cdot y_2 \in \mathbb{Z} \setminus \left\{0\right\}$, we see that $xy$ is of the form $u / v$ where $u \in \mathbb{Z}$ and $v \in \mathbb{Z} \setminus \left\{0\right\}$. Therefore, $xy \in \mathbb{Q}$.
\end{proof}

\item[(b)]
\begin{thm}
If $x$ is a rational number and $y$ is an irrational number, then $x + y$ is an irrational number. If, in addition, $x \neq 0$, then $xy$ is an irrational number.
\end{thm}

Given $x \in \mathbb{Q}$ and $y \notin \mathbb{Q}$, lets suppose that $x + y \in \mathbb{Q}$.

From the definition of the set of rational numbers, $x$ must equal $x_1 / x_2$ for some $x_1 \in \mathbb{Z}$ and for some $x_2 \in \mathbb{Z} \setminus \left\{0\right\}$. It follows that $x + y = x_1 / x_2 + y$
\end{itemize}

\subsection*{Exercise 15(b)}
If $0 < a < b$, show that $1/b < 1/a$.

\subsubsection*{Notes}
Quite abstract.

$0 < a < b \implies \frac{1}{b} < \frac{1}{a}$ don't assume any order properties other than what has been presented.

\underline{Possible approach}:

Given that $0 < a < b$, suppose that $\frac{1}{b} \geq \frac{1}{a}$. Find a contradiction.

\subsubsection*{Solution}
\begin{lem}
Let $x$ be any element of $\mathbb{R}$. If $x > 0$, then $x^{-1} > 0$.
\end{lem}

\begin{proof}
Because $x > 0$ and $x \cdot x^{-1} = 1 > 0$, by Theorem 2.1.10 from the book $x^{-1}$ must be positive.
\end{proof}

\begin{thm}
Let $a, b$ be elements of $\mathbb{R}$. If $0 < a < b$, then $1/b < 1/a$.
\end{thm}

\begin{proof}
From Lemma 3, we see that both $1/a$ and $1/b$ are positive numbers. Knowing this, we can apply part (c) of Theorem 2.1.7 from the book, which states that if $a > b$ and $c > 0$, then $ca > cb$.
\begin{align*}
0 &< a < b \\
0\cdot\left(1/a\right) &< a\cdot\left(1/a\right) < b\cdot\left(1/a\right) \\
0 &< 1 < b / a \\
0\cdot\left(1/b\right) &< 1\cdot\left(1/b\right) < \left(b / a\right)\cdot\left(1 / b\right)\\
0 &< 1/b < 1/a
\end{align*}
And here we conclude that $1 / b < 1 / a$.
\end{proof}

\subsection*{Exercise 18}
Let $a, b \in \mathbb{R}$, and suppose that for every $\epsilon > 0$ we have $a \leq b + \epsilon$. Show that $a \leq b$.

\subsubsection*{Solution}
\begin{thm}
\end{thm}

\section*{Section 2.2 Exercises 5 and 8}

\subsection*{Exercise 5}
If $a < x < b$ and $a < y < b$, show that $\left|x - y\right| < b - a$. interpret this geometrically.

\subsection*{Exercise 8}
Find all values of $x$ that satisfy the following equations:
\begin{itemize}
\item[(a)] $x + 1 = \left|2x - 1\right|$
\item[(b)] $2x - 1 = \left|x - 5\right|$
\end{itemize}

\section*{Section 2.3 Exercises 2, 5, and 11}

\subsection*{Exercise 2}
Let $S_2 \mathrel{\mathop:}= \left\{x \in \mathbb{R} : x > 0\right\}$. Does $S_2$ have lower bounds? Does $S_2$ have upper bounds? Does $\text{inf}S_2$ exist? Does $\text{sup}S_2$ exist? Prove your statements.

\subsection*{Exercise 5}
Find the infimum and supremum, if they exist, of each of the following sets.
\begin{itemize}
\item[(a)] $A \mathrel{\mathop:}= \left\{x \in \mathbb{R} : 2x + 5 > 0\right\}$
\item[(b)] $B \mathrel{\mathop:}= \left\{x \in \mathbb{R} : x + 2 \geq x^2\right\}$
\item[(c)] $C \mathrel{\mathop:}= \left\{x \in \mathbb{R} : x < 1/x\right\}$
\item[(d)] $D \mathrel{\mathop:}= \left\{x \in \mathbb{R} : x^2 - 2x - 5 < 0\right\}$
\end{itemize}

\subsection*{Exercise 11}
Let $S$ be a bounded set in $\mathbb{R}$ and let $S_0$ be a nonempty subset of $S$. Show that $\inf S \leq \inf S_0 \leq \sup S_0 \leq \sup S$.

\section*{Section 2.4 Exercise 7}

\subsection*{Exercise 7}

Let $A$ and $B$ be bounded nonempty subsets of $\mathbb{R}$, and let $A + B \colonequals \left\{a + b : a \in A, b \in B\right\}$. Prove that $\sup\left(A + B\right) = \sup A + \sup B$ and $\inf \left(A + B\right) = \inf A + \inf B$.

\subsubsection*{Solution}
\begin{thm}
Let $A$ and $B$ be bounded nonempty subsets of $\mathbb{R}$, and let $A + B \colonequals \left\{a + b : a \in A, b \in B\right\}$.

Then $\sup\left(A + B\right) = \sup A + \sup B$.
\end{thm}

\begin{proof}
Let $u = \sup A$ and $w = \sup B$. We have $u \geq x$ for all $x \in A$, and $w \geq y$ for all $y \in B$. By adding these inequalities, we get $u + w \geq x + y$ for all $x \in A, y \in B$. Thus, $u + w$ must be an upper bound for $A + B$ and $\sup A + \sup B \geq \sup \left(A + B\right)$.

Let $z = \sup\left(A + B\right)$. Because $z$ is an upper bound of $A + B$, we have $z \geq x + y$ for all $x \in A, y \in B$. It follows that $z - y \geq x$ and that $z - x \geq y$. Notice that $z - y$ and $z - x$ are upper bounds of $A$ and $B$ respectively. Knowing this, rewrite the inequalities as $z - y \geq \sup A$ and $z - x \geq \sup B$. Add these inequalities together to get $2z - x - y \geq \sup A + \sup B$.

Set that inequality aside, and consider that $z \geq x + y$ for all $x \in A, y \in B$. Add $\sup A + \sup B$ to both sides and perform some algebra to make it look more like our previous inequality.
\begin{align*}
\sup A + \sup B + z &\geq x + y + \sup A + \sup B\\
\sup A + \sup B + z - x - y &\geq \sup A + \sup B\\
\intertext{Now subtract this inequality from the inequality that we set aside.}
z - \sup A - \sup B &\geq 0\\
z &\geq \sup A +\sup B \\
\end{align*}
And so we see that $\sup\left(A + B\right) \geq \sup A + \sup B$.

Combine the inequalities and we conclude that $\sup\left(A + B\right) = \sup A + \sup B$.
\end{proof}

To prove that $\inf\left(A + B\right) = \inf A + \inf B$, we follow the same strategy that worked for proving that $\sup\left(A + B\right) = \sup A + \sup B$.

\begin{thm}
Let $A$ and $B$ be bounded nonempty subsets of $\mathbb{R}$, and let $A + B \colonequals \left\{a + b : a \in A, b \in B\right\}$.

Then $\inf\left(A + B\right) = \inf A + \inf B$.
\end{thm}

\begin{proof}
Let $u = \inf A$ and $w = \inf B$. We have $u \leq x$ for all $x \in A$, and $w \leq y$ for all $y \in B$. By adding these inequalities, we get $u + w \leq x + y$ for all $x \in A, y \in B$. Thus, $u + w$ must be a lower bound for $A + B$, and $\inf A + \inf B \leq \inf\left(A + B\right)$.

Let $z = \inf\left(A + B\right)$. Because $z$ is a lower bound of $A + B$, we have $z \leq x + y$ for all $x \in A, y \in B$. It follows that $z - y \leq x$ and that $z - x \leq y$. Notice that $z - y$ and $z - x$ are lower bounds of $A$ and $B$ respectively. Knowing this, rewrite the inequalities as $z - y \leq \inf A$ and $z - x \leq \inf B$. Add these inequalities together to get $2 z - x - y \leq \inf A + \inf B$.

Set that inequality aside, and consider that $z \leq x + y$ for all $x \in A, y \in B$. Add $\inf A + \inf B$ to both sides and perform some algebra to make it look more like our previous inequality.
\begin{align*}
\inf A + \inf B + z &\leq x + y + \inf A + \inf B\\
\inf A + \inf B + z - x - y &\leq \inf A + \inf B\\
\intertext{Now subtract this inequality from the inequality that we set aside.}
z - \inf A - \inf B &\leq 0\\
z &\leq \inf A + \inf B
\end{align*}
And so we see that $\inf\left(A + B\right) \leq \inf A + \inf B$.

Combine the inequalities and we conclude that $\inf\left(A + B\right) = \inf A + \inf B$.
\end{proof}
\end{document}
