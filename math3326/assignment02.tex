\documentclass[12pt]{article}
\special{papersize=8.5in,11in}
\usepackage[utf8]{inputenc}
\usepackage{amssymb,amsmath,amsthm}
\newtheorem*{thm}{Theorem}
\newtheorem{lem}{Lemma}
\usepackage{colonequals}
\usepackage{parskip}
\pagestyle{plain}
\begin{document}
\begin{flushright}
\Large{Jonathan Glines}
\end{flushright}
\section*{Section 2.1 Exercises 5, 8, 15(b), and 18}
\subsection*{Exercise 5}
If $a \ne 0$ and $b \ne 0$, show that $1/\left(ab\right) = \left(1/a\right)\left(1/b\right)$.

%\subsubsection*{Notes}
%Very abstract. Show $\left(ab\right)^{-1} = a^{-1}b^{-1}$. Know \underline{only} that: $\left(ab\right)^{-1}\cdot\left(ab\right) = 1$ ($(ab)(ab)^{-1} = 1$ also)
%\[
%a^{-1} a = 1,\quad
%b^{-1} b = 1
%\]
%Suggestion:
%\begin{enumerate}
%\item Show that $a^{-1}b^{-1}$ has the same property that $(ab)^{-1}$ does, namely $(a^{-1}b^{-1})(ab) = 1$.
%\item Somehow conclude that $(ab)^{-1} = a^{-1}b^{-1}$.
%\end{enumerate}

\subsubsection*{Solution}
\begin{lem}
$\left(ab\right)\left(cd\right) = \left(ac\right)\left(bd\right)$ for all $a, b, c, d \in \mathbb{R}$.
\end{lem}

\begin{proof}
\begin{align*}
\left(ab\right)\left(cd\right) &= \left(a \cdot a^{-1}\right) \left(\left(ab\right)\left(cd\right)\right) & \text{(M4), (M3)}\\
&= \left(\left(a \cdot a^{-1}\right) \left(ab\right)\right)\left(cd\right) & \text{(M2)}\\
&= \left(1 \cdot \left(a \cdot a^{-1}\right) \left(ab\right)\right)\left(cd\right) & \text{(M3)}\\
&= \left(\left(1 \cdot a\right) a^{-1} \left(ab\right)\right)\left(cd\right) & \text{(M2)}\\
&= \left(\left(1 \cdot a\right) \left(a^{-1} a\right)b\right)\left(cd\right) & \text{(M2)}\\
&= \left(\left(1 \cdot a\right) \left(1\right)b\right)\left(cd\right) & \text{(M4)}\\
&= \left(\left(a\right) b\right)\left(cd\right) & \text{(M3)}\\
\end{align*}
\end{proof}

\begin{thm}
$\left(ab\right)^{-1} = a^{-1}b^{-1}$ for all $a, b \in \mathbb{R}$.
\end{thm}

\begin{proof}
Let $a, b$ be elements in $\mathbb{R}$. Notice that because $\left(ab\right)\left(ab\right)^{-1} = 1$, we can apply property (M3) in the following way.
\begin{align*}
a^{-1}b^{-1} &= \left(a^{-1}b^{-1}\right)\left(\left(ab\right)\left(ab\right)^{-1}\right) & \text{(M3)}\\
\intertext{Now we rearange this using property (M2). }
&= \left(\left(a^{-1}b^{-1}\right)\left(ab\right)\right)\left(ab\right)^{-1} & \text{(M2)}
\intertext{From here, apply Lemma 1 and properties (M4) and (M3) to get the following.}
&= \left(\left(a^{-1}a\right)\left(b^{-1}b\right)\right)\left(ab\right)^{-1} & \text{Lemma 1} \\
&= \left(\left(1\right)\left(1\right)\right)\left(ab\right)^{-1} & \text{(M4)} \\
&= \left(1\right)\left(ab\right)^{-1} & \text{(M3)} \\
&= \left(ab\right)^{-1} & \text{(M3)} \\
\end{align*}
Here we conclude that $a^{-1}b^{-1} = \left(ab\right)^{-1}$.
\end{proof}

\subsection*{Exercise 8}
\begin{itemize}
\item[(a)] Show that if $x, y$ are rational numbers, then $x + y$ and $xy$ are rational numbers.
\item[(b)] Prove that if $x$ is a rational number and $y$ is an irrational number, then $x + y$ is an irrational number. If, in addition, $x \neq 0$, then show that $xy$ is an irrational number.
\end{itemize}

%\subsubsection*{Notes}
%\begin{itemize}
%\item[(a)] $x, y \in \mathbb{Q} \implies x + y \in \mathbb{Q} \text{ and } xy \in \mathbb{Q}$
%\item[(b)] $x \in \mathbb{Q}, y \notin \mathbb{Q} \implies x + y \notin \mathbb{Q}$
%\item[(c)] $x \in \mathbb{Q} \setminus \{0\}, y \notin \mathbb{Q} \implies xy \notin \mathbb{Q}$
%\end{itemize}
%Items (b) and (c) can be proved by contradiction.\\
%\underline{Prove by contradiction}\\
%Given that $x \in \mathbb{Q}$ and $y \notin \mathbb{Q}$, assume that $x + y \in \mathbb{Q}$. Find a contradiction.

\subsubsection*{Solution}
\begin{itemize}
\item[(a)]
\begin{lem}
The set $\mathbb{Z} \setminus \left\{0\right\}$ is closed under multiplication.
\end{lem}

\begin{proof}
Let $x, y$ be elements in $\mathbb{Z} \setminus \left\{0\right\}$. Because $\mathbb{Z} \setminus \left\{0\right\}$ is a subset of $\mathbb{Z}$ and $\mathbb{Z}$ is closed under multiplication, the product $xy \in \mathbb{Z}$. The only element in $\mathbb{Z}$ that is not in $\mathbb{Z} \setminus \left\{0\right\}$ is 0, so in order to show that $xy \in \mathbb{Z} \setminus \left\{0\right\}$ it is sufficient to show that $xy \neq 0$.

Suppose to the contrary that $xy = 0$. Let $x_R, y_R$ be elements in $\mathbb{R}$ such that $x_R = x$ and $y_R = y$. We can substitute $x$ and $y$ for $x_R$ and $y_R$ respectively so we can use the reciprocal property of $\mathbb{R}$.
\begin{align*}
xy = \left(x_R\right) \left(y_R\right) &= 0 \\
\left(x_R\right) \left(y_R\right) \left(y_R\right)^{-1} &= 0\cdot\left(y_R\right)^{-1} \\
x_R &= 0 = x
\end{align*}
This contradicts our original premise that $x \in \mathbb{Z} \setminus \left\{0\right\}$. Thus, $xy \neq 0$, and $xy \in \mathbb{Z} \setminus \left\{0\right\}$.
\end{proof}

\begin{thm}
If $x, y$ are rational numbers, then $x + y$ and $xy$ are rational numbers.
\end{thm}

\begin{proof}
Let $x, y$ be rational numbers. From the definition of the set of rational numbers, $x = x_1/x_2$ and $y = y_1/y_2$ for some $x_1, y_1 \in \mathbb{Z}$ and some $x_2, y_2 \in \mathbb{Z} \setminus \left\{0\right\}$.

The number $x + y$ would then equal $\left(x_1y_2 + y_1x_2\right)/\left(x_2 \cdot y_2\right)$. Because $x_1, y_1, x_2, y_2 \in \mathbb{Z}$ and $\mathbb{Z}$ is closed under both addition and multiplication, the numerator $x_1y_2 + y_1x_2 \in \mathbb{Z}$. 
Because $x_2, y_2 \in \mathbb{Z}\setminus\left\{0\right\}$ and $\mathbb{Z}\setminus\left\{0\right\}$ is closed under multiplication (Lemma 2), the divisor $x_2 \cdot y_2 \in \mathbb{Z} \setminus \left\{0\right\}$. Knowing that $x_1y_2 + y_1x_2 \in \mathbb{Z}$ and that $x_1 \cdot y_1 \in \mathbb{Z} \setminus \left\{0\right\}$, we see that $x + y$ is of the form $u / v$ where $u \in \mathbb{Z}$ and $v \in \mathbb{Z} \setminus \left\{0\right\}$. Therefore, $x + y \in \mathbb{Q}$.

The number $xy$ equals $\left(x_1 \cdot y_1\right) / \left(x_2 \cdot y_2\right)$. Because $x_1, y_1 \in \mathbb{Z}$ and $\mathbb{Z}$ is closed under multiplication, the numerator $x_1 \cdot y_1 \in \mathbb{Z}$. Likewise, because $x_2, y_2 \in \mathbb{Z} \setminus \left\{0\right\}$ and $\mathbb{Z} \setminus \left\{0\right\}$ is closed under multiplication (Lemma 2), the denominator $x_2 \cdot y_2 \in \mathbb{Z} \setminus \left\{0\right\}$. Knowing that $x_1 \cdot y_1 \in \mathbb{Z}$ and that $x_2 \cdot y_2 \in \mathbb{Z} \setminus \left\{0\right\}$, we see that $xy$ is of the form $u / v$ where $u \in \mathbb{Z}$ and $v \in \mathbb{Z} \setminus \left\{0\right\}$. Therefore, $xy \in \mathbb{Q}$.
\end{proof}

\item[(b)]
\begin{thm}
If $x$ is a rational number and $y$ is an irrational number, then $x + y$ is an irrational number. If, in addition, $x \neq 0$, then $xy$ is an irrational number.
\end{thm}

\begin{proof}
Given $x \in \mathbb{Q}$ and $y \notin \mathbb{Q}$, lets suppose that $x + y \in \mathbb{Q}$.

Let $z = x + y$. From the definition of the set of rational numbers, $x$ must equal $x_1 / x_2$ and $z$ must equal $z_1 / z_2$ for some $x_1, z_1 \in \mathbb{Z}$ and for some $x_2, z_2 \in \mathbb{Z} \setminus \left\{0\right\}$. It follows that $y = z - x = z_1/z_2 - x_1/x_2 = \left(z_1 x_2 - x_1 z_2\right) / \left(z_2 x_2\right)$. Because $x_1, z_1, x_2, z_2 \in \mathbb{Z}$ and $\mathbb{Z}$ is closed under both addition and multiplication, the numerator $z_1 x_2 - x_1 z_2 \in \mathbb{Z}$. Because $x_2, z_2 \in \mathbb{Z} \setminus \left\{0\right\}$ and $\mathbb{Z} \setminus \left\{0\right\}$ is closed under multiplication (Lemma 2), the denominator $z_2 x_2 \in \mathbb{Z} \setminus \left\{0\right\}$. Knowing that $y = \left(z_1 x_2 - x_1 z_2\right) / \left(z_2 x_2\right)$ and that the numerator and denominator are in sets $\mathbb{Z}$ and $\mathbb{Z} \setminus \left\{0\right\}$ respectievly, we see that $y$ is of the form $u/v$ where $u \in \mathbb{Z}$ and $v \in \mathbb{Z} \setminus \left\{0\right\}$. This suggests that $y \in \mathbb{Q}$, which contradicts our premise that $y \notin \mathbb{Q}$. Thus, $x \in \mathbb{Q}$ and $y \notin \mathbb{Q}$ implies $x + y \notin \mathbb{Q}$.

Given $x \in \mathbb{Q}$ and $y \in \mathbb{Q}$ such that $x \neq 0$, suppose that $xy \in \mathbb{Q}$.

Let $z = xy$. From the definition of the set of rational numbers, $x$ must equal $x_1/x_2$ and $z$ must equal $z_1/z_2$ for some $x_1, z_1 \in \mathbb{Z}$ and for some $x_2, z_2 \in \mathbb{Z} \setminus \left\{0\right\}$. Also note that because $x \neq 0$, in order for $x_1/x_2 \neq 0$ it must be the case that $x_1 \neq 0$ and $x_1 \in \mathbb{Z} \setminus \left\{0\right\}$. It follows that $y = zx^{-1} = \left(z_1/z_2\right)\left(x_2/x_1\right) = \left(z_1 x_2\right)/\left(z_2 x_1\right)$. Because $z_1, x_2 \in \mathbb{Z}$ and $\mathbb{Z}$ is closed under multiplication, the numerator $z_1 x_2 \in \mathbb{Z}$. Because $z_2, x_1 \in \mathbb{Z} \setminus \left\{0\right\}$ and $\mathbb{Z} \setminus \left\{0\right\}$ is closed under multiplication (Lemma 2), the denominator $z_2 x_1 \in \mathbb{Z} \setminus \left\{0\right\}$. Knowing that $y = \left(z_1 x_2\right)/\left(z_2 x_1\right)$ and that the numerator and denominator are in sets $\mathbb{Z}$ and $\mathbb{Z} \setminus \left\{0\right\}$ respectively, we see that $y$ is of the form $u/v$ where $u \in \mathbb{Z}$ and $v \in \mathbb{Z} \setminus \left\{0\right\}$. This suggests that $y \in \mathbb{Q}$, which contradicts our premise that $y \notin \mathbb{Q}$. Thus, $x \in \mathbb{Q}$ and $y \notin \mathbb{Q}$ with $x \neq 0$ implies $xy \notin \mathbb{Q}$.
\end{proof}
\end{itemize}

\subsection*{Exercise 15(b)}
If $0 < a < b$, show that $1/b < 1/a$.

%\subsubsection*{Notes}
%Quite abstract.
%
%$0 < a < b \implies \frac{1}{b} < \frac{1}{a}$ don't assume any order properties other than what has been presented.
%
%\underline{Possible approach}:
%
%Given that $0 < a < b$, suppose that $\frac{1}{b} \geq \frac{1}{a}$. Find a contradiction.

\subsubsection*{Solution}
\begin{lem}
Let $x$ be any element of $\mathbb{R}$. If $x > 0$, then $x^{-1} > 0$.
\end{lem}

\begin{proof}
Because $x > 0$ and $x \cdot x^{-1} = 1 > 0$, by Theorem 2.1.10 from the book $x^{-1}$ must be positive.
\end{proof}

\begin{thm}
Let $a, b$ be elements of $\mathbb{R}$. If $0 < a < b$, then $1/b < 1/a$.
\end{thm}

\begin{proof}
From Lemma 3, we see that both $1/a$ and $1/b$ are positive numbers. Knowing this, we can apply part (c) of Theorem 2.1.7 from the book, which states that if $a > b$ and $c > 0$, then $ca > cb$.
\begin{align*}
0 &< a < b \\
0\cdot\left(1/a\right) &< a\cdot\left(1/a\right) < b\cdot\left(1/a\right) \\
0 &< 1 < b / a \\
0\cdot\left(1/b\right) &< 1\cdot\left(1/b\right) < \left(b / a\right)\cdot\left(1 / b\right)\\
0 &< 1/b < 1/a
\end{align*}
And here we conclude that $1 / b < 1 / a$.
\end{proof}

\subsection*{Exercise 18}
Let $a, b \in \mathbb{R}$, and suppose that for every $\varepsilon > 0$ we have $a \leq b + \varepsilon$. Show that $a \leq b$.

\subsubsection*{Solution}
\begin{thm}
Given $a, b \in \mathbb{R}$ such that for every $\varepsilon > 0$ we have $a \leq b + \varepsilon$, $a \leq b$.
\end{thm}

By subtracting $\varepsilon$ from both sides of the given inequality, we find that $a - \varepsilon \leq b$.

Because $\varepsilon > 0$, with some algebra it follows that $a - \varepsilon < a$. Combine this inequality with our earlier inequality and we have.

\section*{Section 2.2 Exercises 5 and 8}

\subsection*{Exercise 5}
If $a < x < b$ and $a < y < b$, show that $\left|x - y\right| < b - a$. Interpret this geometrically.

\subsubsection*{Solution}
\begin{thm}
If $a < x < b$ and $a < y < b$, then $\left| x - y \right| < b - a$.
\end{thm}

The inequality $\left| x - y \right| < b - a$ is equivilant to saying that $x - y > 0 \implies x - y < b - a$ and that $x - y < 0 \implies -\left(x - y\right) < b - a$.

\subsection*{Exercise 8}
Find all values of $x$ that satisfy the following equations:
\begin{itemize}
\item[(a)] $x + 1 = \left|2x - 1\right|$
\item[(b)] $2x - 1 = \left|x - 5\right|$
\end{itemize}

\subsubsection*{Solution}
\begin{itemize}
\item[(a)]
\item[(b)]
\end{itemize}

\section*{Section 2.3 Exercises 2, 5, and 11}

\subsection*{Exercise 2}
Let $S_2 \mathrel{\mathop:}= \left\{x \in \mathbb{R} : x > 0\right\}$. Does $S_2$ have lower bounds? Does $S_2$ have upper bounds? Does $\text{inf}S_2$ exist? Does $\text{sup}S_2$ exist? Prove your statements.

\subsubsection*{Solution}
$S_2$ does have lower bounds. The number -1 is a lower bound for $S_2$, because for every element $x \in S_2$, $x$ is greater than $0$ and also clearly greater than $-1$.

$S_2$ does not have upper bounds. Let $S_N$ be the set $\left\{n \in \mathbb{N} : n \in S_2\right\}$. The number $1$ is in the set $S_N$ because $1$ is in the set $S_2$. Let $n_0$ be any number in $S_N$. Because $n_0$ is an element of $S_N$, $n_0$ must be greater than $0$. We see immediately that $n_0 + 1$ is also greater than $0$, which means $n_0 + 1 \in S_2$ and thus $n_0 + 1 \in S_N$. By Mathematical Induction, $S_N = \mathbb{N}$, and we see that $\mathbb{N} \subset S_2$. Because $\mathbb{N}$ has no upper bounds, we conclude that $S_2$ has no upper bounds.

$\inf S_2$ does exist. Let $\bar{S}_2$ be the set $\left\{x : -x \in S_2\right\}$. Any lower bound $l \in S_2$ corresponds to an upper bound $-l \in \bar{S}_2$, because for every $x \in S_2$, we have $l \leq x$ and $-l \geq -x$. Because $\bar{S}_2$ is bounded above, the Completeness Property of $\mathbb{R}$ tells us that $\sup \bar{S}_2$ exists. Let $u$ be any upper bound for $\bar{S}_2$. It follows that for any $-x \in \bar{S}_2$ we have $u \geq \sup \bar{S}_2 \geq -x$ and also $-u \leq -\sup\bar{S}_2 \leq x$. FIXME

$\sup S_2$ does not exist, because $S_2$ does not have any upper bounds.

\subsection*{Exercise 5}
Find the infimum and supremum, if they exist, of each of the following sets.
\begin{itemize}
\item[(a)] $A \mathrel{\mathop:}= \left\{x \in \mathbb{R} : 2x + 5 > 0\right\}$
\item[(b)] $B \mathrel{\mathop:}= \left\{x \in \mathbb{R} : x + 2 \geq x^2\right\}$
\item[(c)] $C \mathrel{\mathop:}= \left\{x \in \mathbb{R} : x < 1/x\right\}$
\item[(d)] $D \mathrel{\mathop:}= \left\{x \in \mathbb{R} : x^2 - 2x - 5 < 0\right\}$
\end{itemize}

\subsubsection*{Solution}
\begin{itemize}
\item[(a)]
\begin{align*}
2x + 5 &> 0\\
2x &> -5\\
x &> -5/2\\
\end{align*}
From this algebra, we see that $\left\{x \in \mathbb{R} : 2x + 5 > 0\right\} = \left\{x \in \mathbb{R} : x > -5/2\right\}$, and conjecture that $-5/2$ is the infimum of $A$.

For any number $x \in A$, we see that $x > -5/2$. Clearly, $-5/2$ is less than any $x \in A$, and $-5/2$ is lower bound for $A$.

To see that $-5/2$ is the greatest lower bound for $A$, let $t$ be any number greater than $-5/2$. From the way the set $A$ is defined, we see that any such $t$ satisfies the inequality $x > -5/2$, which means that $t \in A$.

Thus, $\inf A = -5/2$.
\item[(b)]
$B \mathrel{\mathop:}= \left\{x \in \mathbb{R} : x + 2 \geq x^2\right\}$
\begin{align*}
x + 2 &\geq x^2 \\
-x^2 + x + 2 &\geq 0 \\
x^2 - x - 2 &\leq 0 \\
\left(x + 1\right)\left(x - 2\right) &\leq 0 \\
\end{align*}
$\left(x + 1\right)\left(x - 2\right)$ has roots at $-1$ and $2$, so consider some values in that range and conjecture that $\inf B = -1$ and $\sup B = 2$.

To prove that $-1 = \inf B$, we first check that -1 is a lower bound for $B$. All elements $x$ in $B$ satisfy the inequality $-x^2 + x \geq -2$. Any element $y < -1$ will not satisfy that inequality because $-\left(-1\right)^2 + \left(-1\right) = -2$ and the expression $-y^2 + y$ only decreases for decreasing values of $y$. Thus, no elements in $B$ are less than $-1$, and $-1$ is a lower bound for $B$.
\item[(c)]
First use some algebra to simplify the inequality.
\begin{align*}
x &< 1 / x\\
x^2 &< 1\\
-1 < x &< 1\\
\end{align*}
We see that the set $C \mathrel{\mathop:}= \left\{x \in \mathbb{R} : x < 1/x\right\}$ equals the set $\left\{x \in \mathbb{R} : -1 < x < 1\right\}$ and that clearly $-1$ and $1$ are lower and upper bounds of $C$ respectively; any element $x \in C$ must be greater than $-1$, and less than $1$.

To prove that $-1$ is the greatest lower bound of $C$, consider any element $t$ directly above $-1$ such that $0 > t > -1$. Then there exists at least one element $u \in C$ such that $t > u$, one example being the element halfway between $-1$ and $t$. Thus, $t$ is not a lower bound, and since $t$ is an arbitrary number directly above $-1$, we conclude that $\inf C = -1$.
%$-1 + \frac{ \left| t - \left(-1\right) \right| }{2} = -1 + \frac{t + 1}{2} = t/2 - 1/2$.

Similarly, to prove that $1$ is the lowest upper bound of $C$, consider any element $t$ directly below $1$ such that $0 < t < 1$. Then there exists at least one element $u \in C$ such that $t < u$, one example being the element halfway between $t$ and $1$. Thus, $t$ is not an upper bound, and since $t$ is an arbitrary number directly below $1$, we conclude that $\sup C = 1$.
\item[(d)]$D \mathrel{\mathop:}= \left\{x \in \mathbb{R} : x^2 - 2x - 5 < 0\right\}$
\end{itemize}
\subsection*{Exercise 11}
Let $S$ be a bounded set in $\mathbb{R}$ and let $S_0$ be a nonempty subset of $S$. Show that $\inf S \leq \inf S_0 \leq \sup S_0 \leq \sup S$.

\subsubsection*{Solution}
\begin{thm}
Let $S$ be a bounded set in $\mathbb{R}$ and let $S_0$ be a nonempty subset of $S$. Then $\inf S \leq \inf S_0 \leq \sup S_0 \leq \sup S$.
\end{thm}

\section*{Section 2.4 Exercise 7}

\subsection*{Exercise 7}

Let $A$ and $B$ be bounded nonempty subsets of $\mathbb{R}$, and let $A + B \colonequals \left\{a + b : a \in A, b \in B\right\}$. Prove that $\sup\left(A + B\right) = \sup A + \sup B$ and $\inf \left(A + B\right) = \inf A + \inf B$.

\subsubsection*{Solution}
\begin{thm}
Let $A$ and $B$ be bounded nonempty subsets of $\mathbb{R}$, and let $A + B \colonequals \left\{a + b : a \in A, b \in B\right\}$.

Then $\sup\left(A + B\right) = \sup A + \sup B$.
\end{thm}

\begin{proof}
Let $u = \sup A$ and $w = \sup B$. We have $u \geq x$ for all $x \in A$, and $w \geq y$ for all $y \in B$. By adding these inequalities, we get $u + w \geq x + y$ for all $x \in A, y \in B$. Thus, $u + w$ must be an upper bound for $A + B$ and $\sup A + \sup B \geq \sup \left(A + B\right)$.

Let $z = \sup\left(A + B\right)$. Because $z$ is an upper bound of $A + B$, we have $z \geq x + y$ for all $x \in A, y \in B$. It follows that $z - y \geq x$ and that $z - x \geq y$. Notice that $z - y$ and $z - x$ are upper bounds of $A$ and $B$ respectively. Knowing this, rewrite the inequalities as $z - y \geq \sup A$ and $z - x \geq \sup B$. Add these inequalities together to get $2z - x - y \geq \sup A + \sup B$.

Set that inequality aside, and consider that $z \geq x + y$ for all $x \in A, y \in B$. Add $\sup A + \sup B$ to both sides and perform some algebra to make it look more like our previous inequality.
\begin{align*}
\sup A + \sup B + z &\geq x + y + \sup A + \sup B\\
\sup A + \sup B + z - x - y &\geq \sup A + \sup B\\
\intertext{Now subtract this inequality from the inequality that we set aside.}
z - \sup A - \sup B &\geq 0\\
z &\geq \sup A +\sup B \\
\end{align*}
And so we see that $\sup\left(A + B\right) \geq \sup A + \sup B$.

Combine the inequalities and we conclude that $\sup\left(A + B\right) = \sup A + \sup B$.
\end{proof}

To prove that $\inf\left(A + B\right) = \inf A + \inf B$, we follow the same strategy that worked for proving that $\sup\left(A + B\right) = \sup A + \sup B$.

\begin{thm}
Let $A$ and $B$ be bounded nonempty subsets of $\mathbb{R}$, and let $A + B \colonequals \left\{a + b : a \in A, b \in B\right\}$.

Then $\inf\left(A + B\right) = \inf A + \inf B$.
\end{thm}

\begin{proof}
Let $u = \inf A$ and $w = \inf B$. We have $u \leq x$ for all $x \in A$, and $w \leq y$ for all $y \in B$. By adding these inequalities, we get $u + w \leq x + y$ for all $x \in A, y \in B$. Thus, $u + w$ must be a lower bound for $A + B$, and $\inf A + \inf B \leq \inf\left(A + B\right)$.

Let $z = \inf\left(A + B\right)$. Because $z$ is a lower bound of $A + B$, we have $z \leq x + y$ for all $x \in A, y \in B$. It follows that $z - y \leq x$ and that $z - x \leq y$. Notice that $z - y$ and $z - x$ are lower bounds of $A$ and $B$ respectively. Knowing this, rewrite the inequalities as $z - y \leq \inf A$ and $z - x \leq \inf B$. Add these inequalities together to get $2 z - x - y \leq \inf A + \inf B$.

Set that inequality aside, and consider that $z \leq x + y$ for all $x \in A, y \in B$. Add $\inf A + \inf B$ to both sides and perform some algebra to make it look more like our previous inequality.
\begin{align*}
\inf A + \inf B + z &\leq x + y + \inf A + \inf B\\
\inf A + \inf B + z - x - y &\leq \inf A + \inf B\\
\intertext{Now subtract this inequality from the inequality that we set aside.}
z - \inf A - \inf B &\leq 0\\
z &\leq \inf A + \inf B
\end{align*}
And so we see that $\inf\left(A + B\right) \leq \inf A + \inf B$.

Combine the inequalities and we conclude that $\inf\left(A + B\right) = \inf A + \inf B$.
\end{proof}
\end{document}
