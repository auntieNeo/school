\documentclass[12pt]{article}
\special{papersize=8.5in,11in}
\usepackage[utf8]{inputenc}
\usepackage{amssymb,amsmath}
\usepackage{colonequals}
\pagestyle{plain}
\begin{document}
\begin{flushright}
\Large{Jonathan Glines}
\end{flushright}
\begin{flushleft}
\section*{Section 2.1 Exercises 5, 8, 15(b), and 18}
\subsection*{Exercise 5}
If $a \ne 0$ and $b \ne 0$, show that $1/\left(ab\right) = \left(1/a\right)\left(1/b\right)$.

\subsubsection*{Notes}
Very abstract. Show $\left(ab\right)^{-1} = a^{-1}b^{-1}$. Know \underline{only} that: $\left(ab\right)^{-1}\cdot\left(ab\right) = 1$ ($(ab)(ab)^{-1} = 1$ also)
\[
a^{-1} a = 1,\quad
b^{-1} b = 1
\]
Suggestion:
\begin{enumerate}
\item Show that $a^{-1}b^{-1}$ has the same property that $(ab)^{-1}$ does, namely $(a^{-1}b^{-1})(ab) = 1$.
\item Somehow conclude that $(ab)^{-1} = a^{-1}b^{-1}$.
\end{enumerate}

\subsection*{Exercise 8}
\begin{itemize}
\item[(a)] Show that if $x, y$ are rational numbers, then $x + y$ and $xy$ are rational numbers.
\item[(b)] Prove that if $x$ is a rational number and $y$ is an irrational number, then $x + y$ is an irrational number. If, in addition, $x \neq 0$, then show that $xy$ is an irrational number.
\end{itemize}

\subsubsection*{Notes}
\begin{itemize}
\item[(a)] $x, y \in \mathbb{Q} \implies x + y \in \mathbb{Q} \text{ and } xy \in \mathbb{Q}$
\item[(b)] $x \in \mathbb{Q}, y \notin \mathbb{Q} \implies x + y \notin \mathbb{Q}$
\item[(c)] $x \in \mathbb{Q} \setminus \{0\}, y \notin \mathbb{Q} \implies xy \notin \mathbb{Q}$
\end{itemize}
Items (b) and (c) can be proved by contradiction.\\
\underline{Prove by contradiction}\\
Given that $x \in \mathbb{Q}$ and $y \notin \mathbb{Q}$, assume that $x + y \in \mathbb{Q}$. Find a contradiction.

\subsubsection*{Solution}
\begin{itemize}
\item[(a)]
\item[(b)]
\end{itemize}

\subsection*{Exercise 15(b)}
If $0 < a < b$, show that $1/b < 1/a$.

\subsubsection*{Notes}
Quite abstract.

$0 < a < b \implies \frac{1}{b} < \frac{1}{a}$ don't assume any order properties other than what has been presented.

\underline{Possible approach}:

Given that $0 < a < b$, suppose that $\frac{1}{b} \geq \frac{1}{a}$. Find a contradiction.

\subsection*{Exercise 18}
Let $a, b \in \mathbb{R}$, and suppose that for every $\epsilon > 0$ we have $a \leq b + \epsilon$. Show that $a \leq b$.

\section*{Section 2.2 Exercises 5 and 8}

\subsection*{Exercise 5}
If $a < x < b$ and $a < y < b$, show that $\left|x - y\right| < b - a$. interpret this geometrically.

\subsection*{Exercise 8}
Find all values of $x$ that satisfy the following equations:
\begin{itemize}
\item[(a)] $x + 1 = \left|2x - 1\right|$
\item[(b)] $2x - 1 = \left|x - 5\right|$
\end{itemize}

\section*{Section 2.3 Exercises 2, 5, and 11}

\subsection*{Exercise 2}
Let $S_2 \mathrel{\mathop:}= \left\{x \in \mathbb{R} : x > 0\right\}$. Does $S_2$ have lower bounds? Does $S_2$ have upper bounds? Does $\text{inf}S_2$ exist? Does $\text{sup}S_2$ exist? Prove your statements.

\subsection*{Exercise 5}
Find the infimum and supremum, if they exist, of each of the following sets.
\begin{itemize}
\item[(a)] $A \mathrel{\mathop:}= \left\{x \in \mathbb{R} : 2x + 5 > 0\right\}$
\item[(b)] $B \mathrel{\mathop:}= \left\{x \in \mathbb{R} : x + 2 \geq x^2\right\}$
\item[(c)] $C \mathrel{\mathop:}= \left\{x \in \mathbb{R} : x < 1/x\right\}$
\item[(d)] $D \mathrel{\mathop:}= \left\{x \in \mathbb{R} : x^2 - 2x - 5 < 0\right\}$
\end{itemize}

\subsection*{Exercise 11}
Let $S$ be a bounded set in $\mathbb{R}$ and let $S_0$ be a nonempty subset of $S$. Show that $\inf S \leq \inf S_0 \leq \sup S_0 \leq \sup S$.

\section*{Section 2.4 Exercise 7}

\subsection*{Exercise 7}

Let $A$ and $B$ be bounded nonempty subsets of $\mathbb{R}$, and let $A + B \colonequals \left\{a + b : a \in A, b \in B\right\}$. Prove that $\sup\left(A + B\right) = \sup A + \sup B$ and $\inf \left(A + B\right) = \inf A + \inf B$.

\end{flushleft}
\end{document}
