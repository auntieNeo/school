\documentclass[12pt]{article}
\special{papersize=8.5in,11in}
\usepackage[utf8]{inputenc}
\usepackage{amssymb,amsmath}
\usepackage[pdftex]{graphicx}
\usepackage{graphviz}
\pagestyle{plain}
\begin{document}
\begin{flushright}
{
\Large Jonathan Glines\\
\Large CS 4481\\
\Large Assignment 2\\
}
\end{flushright}
\subsection*{1. Exercise 1.9}
An alternative to the method fo porting a compiler described in Section 1.6 and Figure 1.3 is to use an interpreter for the intermediate code produced by the compiler and to do away with a back end altogether. Such a method is used by the Pascal P-system, which includes a Pascal compiler that produces P-code, a kind of assembly code for a ``generic" stack machine, and a P-code interpreter that simulates the execution of the P-code. Both the Pascal compiler and the P-code interpreter are written in P-code.
\begin{itemize}
\item[a.] Describe the steps needed to obtain a working Pascal compiler on an arbitrary machine, given a Pascal P-system.
\subsubsection*{Answer}
Given a Pascal P-system, the only thing that needs to be done to get a Pascal compiler on this machine is to write a P-code interpreter. The interpreter can be written in machine code, or it can be written in any other language with existing compilers or assemblers for this machine.

\item[b.] Describe the steps needed to obtain a working native-code compiler from your system in (a)(i.e., a compiler that produces executible code for the host machine, rather than using the P-code interpreter.)
\subsubsection*{Answer}
\begin{enumerate}
\item To write a native-code compiler for the given machine $B$, we would first rewrite the source code for an existing compiler $C_1$ that runs on machine $A$ to compile code for machine $B$. We'll call the source for this new compiler $S$. We can even rewrite our P-code compiler and use a P-code interpreter for machine $A$.
\item Now we compile $S$ using $C_1$ (which already runs on machine $A$) to obtain a retargeted compiler $C_2$ which runs on machine $A$ but generates code that runs on machine $B$.
\item Finally we compile $S$ again, but this time using $C_2$ to obtain a compiler $C_3$ that both runs on machine $B$ and generates code for machine $B$.
\end{enumerate}
\end{itemize}

\subsection*{2. Exercise 2.1 Parts (a), (e), (f), and (i)}
\begin{itemize}
\item[a.] All strings of lowercase letters that begin and end in {\tt a}.
\begin{verbatim}((a[a-z]*a)|a)\end{verbatim}
\item[e.] All strings of digits such that all the {\tt 2}'s occur before all the {\tt 9}'s.
\begin{verbatim}(([0-8]*[0-13-9]*)|\d)\end{verbatim}
\item[f.] All strings of {\tt a}'s and {\tt b}'s that contain no three consecutive {\tt b}'s.

Regular expressions cannot conut, so this cannot be expressed as a regular expression.
\item[i.] All strings of {\tt a}'s and {\tt b}'s that contain exactly as many {\tt a}'s as {\tt b}'s.

Again, regular expressions cannot conut. With this set of strings, we would need to count the number of {\tt a}'s and {\tt b}'s, which is impossible with regular expresisons.
\begin{verbatim}\end{verbatim}
\end{itemize}

\subsection*{3.}
For each part of Problem 2.1 such that it is possible to create a regular expression, design an NFA for the language.
\begin{itemize}
\item[a.] \begin{verbatim}a[a-z]a\end{verbatim}
\digraph[scale=0.5]{aNFA}{
node [shape=circle]
start [style="invis"]
start -> s0 [label="start"]
s0 -> s1 [label="a"]
s1 -> s2 [label="[a-z]"]
s2 -> s2 [label="[a-z]"]
s2 -> s3 [label="a"]
s1 -> s3 [label="a"]
s5 [shape=doublecircle]
s3 -> s5 [label="&epsilon;"]
s0 -> s4 [label="a"]
s4 -> s5 [label="&epsilon;"]
}
\item[b.] \begin{verbatim}(([0-8]*[0-13-9]*)|\d)\end{verbatim}
\digraph[scale=0.5]{bNFA}{
node [shape=circle]
start [style="invis"]
start -> s0 [label="start"]
s0 -> s1 [label="&epsilon;"]
s1 -> s2 [label="[0-8]"]
s2 -> s2 [label="[0-8]"]
s2 -> s3 [label="&epsilon;"]
s1 -> s3 [label="&epsilon;"]
s3 -> s4 [label="&epsilon;"]
s4 -> s5 [label="[0-13-9]"]
s5 -> s5 [label="[0-13-9]"]
s5 -> s6 [label="&epsilon;"]
s4 -> s6 [label="&epsilon;"]
s9 [shape=doublecircle]
s6 -> s9 [label="&epsilon;"]
s0 -> s7 [label="&epsilon;"]
s7 -> s8 [label="decimal digit"]
s8 -> s9 [label="&epsilon;"]
}

\end{itemize}
%And here's a graph based on Ken Thompson's paper {\it Regular Expression Search Algorithm}.
%\digraph[scale=0.5]{kenNFA}{
%start [style="invis"]
%n0 [label="+"]
%start -> n0
%n1 [label="a"]
%n0 -> n1
%n2 [label="+"]
%n1 -> n2
%n3 [label="[a-z]"]
%n2 -> n3
%n3 -> n2
%n4 [label="a"]
%n2 -> n4
%end [style="invis"]
%n4 -> end
%n5 [label="a"]
%n0 -> n5
%n5 -> end
%}

\subsection*{4.}
For each NFA in the previous problem, convert the NFA into a DFA using the procedure shown in class.
\begin{itemize}
\item[a.] \begin{verbatim}a[a-z]a\end{verbatim}
\digraph[scale=0.5]{aDFA}{
node [shape=circle]
start [style="invis"]
q0 [label="{s0}"]
start -> q0 [label="start"]
q1 [label="{s1, s5}", shape=doublecircle]
q0 -> q1 [label="a"]
q2 [label="{s2}"]
q1 -> q2 [label="[b-z]"]
q2 -> q2 [label="[b-z]"]
q3 [label="{s5, s2}", shape=doublecircle]
q2 -> q3 [label="a"]
q3 -> q3 [label="a"]
q1 -> q3 [label="a"]
q3 -> q2 [label="[b-z]"]
}
\item[b.] \begin{verbatim}(([0-8]*[0-13-9]*)|\d)\end{verbatim}
\digraph[scale=0.5]{bDFA}{
node [shape=circle]
start [style="invis"]
q0 [shape=doublecircle, label="{s1, s4, s7, s9}"]
q1 [shape=doublecircle, label="{s5, s9}"]
q2 [shape=doublecircle, label="{s2, s5, s9}"]
q3 [shape=doublecircle, label="{s2, s9}"]
q4 [shape=doublecircle, label="{s9}"]
start -> q0 [label="start"]
q0 -> q1 [label="9"]
q0 -> q2 [label="[0-13-8]"]
q0 -> q3 [label="2"]
q1 -> q1 [label="[0-13-9]"]
q1 -> q4 [label="2"]
q2 -> q1 [label="9"]
q2 -> q3 [label="2"]
q2 -> q4 [label="[0-13-8]"]
q3 -> q1 [label="9"]
q3 -> q2 [label="[0-13-8]"]
q3 -> q3 [label="2"]
}
\end{itemize}

\end{document}
