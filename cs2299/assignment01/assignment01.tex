\documentclass[12pt]{article}
\special{papersize=8.5in,11in}
\usepackage[utf8]{inputenc}
\usepackage{amssymb,amsmath}
\pagestyle{plain}
\begin{document}
\begin{flushright}
\Large{Jonathan Glines}
\end{flushright}
\section*{Assignment 1}
\subsection*{Exercise 1}
According to the text and presentaton, what are the classes of computing applications (four)? Provide a one sentence description of each.
\subsubsection*{Solution}
\begin{itemize}
\item[Desktop Computers] Computers meant to be used by a single user for tasks that require graphics and input, such as word processing, web browsing, and games.
\item[Servers] Computers on the network meant to be used by multiple, simultaneous users for consolidating data and facilitating its exchange.
\item[Supercomputers] Groups of servers used for scientific and engineering computations that achieve high computation throughput by providing many processors.
\item[Embedded computers (processors)] A computer within a device designed to fill a specific application, typically with low power, low cost, and marginal performance.
\end{itemize}

\subsection*{Exercise 2}
Name five devices with embedded computers that you have used?
\subsubsection*{Solution}
\begin{itemize}
\item Microwave
\item Wrist Watch
\item Network Card
\item Ethernet Switch
\item USB Keyboard
\end{itemize}

\subsection*{Exercise 3}
According to the class text and presentation, what are the five classic components of a computer?
\subsubsection*{Solution}
\begin{itemize}
\item Input
\item Output
\item Memory
\item Datapath
\item Controlpath
\end{itemize}

\subsection*{Exercise 4}
What is an Instruction Set Architecture (ISA)?
\subsubsection*{Solution}
An ISA is an interface that specifies the machine instructions that a particular machine provides to the lowest level of software, written in machine code, that run on that machine.

\subsection*{Exercise 5}
Define the following terms:
\begin{itemize}
\item[response time] Something
\end{itemize}
\end{document}
