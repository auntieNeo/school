\documentclass[12pt]{article}
\special{papersize=8.5in,11in}
\usepackage[utf8]{inputenc}
\usepackage{amssymb,amsmath}
\usepackage{mathtools}
\usepackage{listings}
\usepackage{array}
\newcolumntype{C}[1]{>{\centering\let\newline\\\arraybackslash\hspace{0pt}}m{#1}}
\pagestyle{plain}
\begin{document}
\begin{flushright}
{\Large Jonathan Glines \\
CS 2299 \\
Assignment 1 \\
}
\end{flushright}
\lstset{breakatwhitespace=true,breaklines=true,basicstyle={\small\ttfamily}}
\section*{Assignment 4}
\subsection*{Exercise 1}
For the MIPS assembly codes below, what is the corresponding C statement. Assume that the variables f, g, h, i, and j are assigned to {\tt \$s0}, {\tt \$s1}, {\tt \$s2}, {\tt \$s3}, and {\tt \$s4} respectively. Assume that the base address of the arrays A and B are in registers {\tt \$s6} and {\tt \$s7} respectively.
\begin{itemize}
\item[(a)]
\lstinputlisting{exercise1a.s}

\item[(b)]
\lstinputlisting{exercise1b.s}
\end{itemize}
\subsubsection*{Solution}
\begin{itemize}
\item[(a)] \lstinputlisting{exercise1a.c}

\item[(b)] \lstinputlisting{exercise01b.c}
\end{itemize}

\subsection*{Exercise 2}
For the MIPS assembly code below:
\begin{itemize}
\item[(a)] What is the correspnoding C statement? Assume that the variables f, g, and h are assigned to {\tt \$s0}, {\tt \$s1}, and {\tt \$s2} respectively.
\item[(b)] Find the value of {\tt \$s0} at the end of the assembly code. Assume that the registers {\tt \$s0}, {\tt \$s1}, and {\tt \$s2} contain the values {\tt 0x0000000a}, {\tt 0x00000014}, and {\tt 0x00000028} respectively.
\end{itemize}
\lstinputlisting{exercise2.s}
\subsubsection*{Solution}
\begin{itemize}
\item[(a)] \lstinputlisting[language=C]{exercise2.c}
\item[(b)] {\tt 0xa - 0x28 = -30}
\end{itemize}

\subsection*{Exercise 3}
For the C statements below, what is the corresponding MIPS assembly code? Use a minimal number of MIPS assembly instructions.
\begin{itemize}
\item[(a)] {\tt f = f + g + h + i + j + 2;}
\item[(b)] {\tt f = g - (f + 5);}
\item[(c)] {\tt f = -g + h + B[1];}
\item[(d)] {\tt f = A[B[g] + 1];}
\end{itemize}
\subsubsection*{Solution}
\begin{itemize}
\item[(a)] \lstinputlisting{exercise03a.s}

\item[(b)] \lstinputlisting{exercise03b.s}

\item[(c)] \lstinputlisting{exercise03c.s}

\item[(d)] \lstinputlisting{exercise03d.s}
\end{itemize}

\subsection*{Exercise 5}
What is the shortest sequence of MIPS instructions that extracts a field for the constant values of bits 7-21 (inclusive) from register {\tt\$t0} and places it in the lower portion of register {\tt\$t3} (zero filled otherwise).
\subsubsection*{Solution}
Using the zero-fill property of shifts, this can be done in two instructions:
\lstinputlisting{exercise05.s}

\subsection*{Exercise 6}
What is the shortest sequence of MIPS instructions that extracts a field for the constant values of bits 23-30 (inclusive) from register {\tt\$t0} and places it in the lower portion of register {\tt\$t2} (zero filled otherwise).
\subsubsection*{Solution}
\lstinputlisting{exercise06.s}

\subsection*{Exercise 8}
Describe what the following MIPS code computes. Assume that {\tt \$a0} and {\tt \$a1} are used for the input and both initially contain the integers x and y respectively. Assume the {\tt\$v0} is used for the output.
\lstinputlisting{exercise08.s}
\subsubsection*{Solution}
This code uses a loop to add the value of $x$ to a temporary value {\tt\$t0} (which starts at zero) a total of $y$ times (assuming $y$ is positive). As a side effect, the value of $y$ is decremented until it is equal to zero. In the final step, the temporary {\tt \$t0} is incremented by 10, and the output value {\tt \$v0} is set to {\tt \$t0 + 0} or simply {\tt\$t0}. Assuming $y$ is positive and no overflow, the output is $x \times y + 10$.

\subsection*{Exercise 9}
List the six possible fields (include the name and meaning) in a MIPS instruction?
\subsubsection*{Solution}
\begin{itemize}
\item[op] 6-bit opcode that specifies the operation
\item[rs] 5-bit GPR (General Purpose Register) address of the source operand
\item[rt] 5-bit GPR address of the second source operand
\item[rd] 5-bit GPR address of the result's destination
\item[shamt] 5-bit shift amount (for shift instructions)
\item[funct] 6-bit function code agumenting the opcode
\end{itemize}

\subsection*{Exercise 10}
What are the three basic instruction formats? Show the layout/fields for each.
\subsubsection*{Solution}
\begin{itemize}
\item[R format]
\begin{tabular}{| C{2cm} | C{2cm} | C{2cm} | C{2cm} | C{2cm} | C{2cm} |}
\hline
op & rs & rt & rd & shamt & funct \\
\hline
\end{tabular}
\item[I format]
\begin{tabular}{| C{2cm} | C{2cm} | C{2cm} | C{6.9cm} |}
\hline
op & rs & rt & immediate \\
\hline
\end{tabular}
\item[J format]
\begin{tabular}{| C{2cm} | C{11.8cm} |}
\hline
op & jump target \\
\hline
\end{tabular}
\end{itemize}
% Answer is on slide 8

\subsection*{Exercise 14}
For the loop written in MIPS assembly below:
\begin{itemize}
\item[(a)] Assume that the register {\tt\$t1} is initialized to the value 10. What is the value in register {\tt\$s2} assuming the {\tt\$s2} is initially zero?
\item[(b)] Write the equivalent C code routine. Assume the registers {\tt\$s1}, {\tt\$s2}, {\tt\$t1}, and {\tt\$t2} Are integers A, B, i, and temp, respectively..
\item[(c)] Assume that the register {\tt\$t1} is initiated to the value $N$. How many MIPS instructions are executed?
\end{itemize}
\lstinputlisting{exercise14.s}
\subsubsection*{Solution}
Note: There's no {\tt subi} instruction, according to both the slides as well as the MIPS32 documentation.
\begin{itemize}
\item[(a)] The register {\tt\$s2} would be 22.
\item[(b)] \lstinputlisting{exercise14.c}
\item[(c)] The loop, which consists of 5 instructions, is executed $N$ times. This makes for $N \times 5$ instructions, plus the 2 instructions at the end to exit the loop, so a total of $\boxed{N \times 5 + 2}$ instructions.
\end{itemize}
\end{document}
