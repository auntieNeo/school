\documentclass[12pt,letterpaper]{article}
\usepackage{ifpdf}
\usepackage{mla}
\begin{document}
\begin{mla}{Jonathan}{Glines}{Professor Pelletti}{Philosophy 1101}{26 September 2011}{Gottfried Leibniz's \textit{Discourse on Metaphysics} and the Implications of a Perfect World}

%1) Would Monotheistic beliefs allow one to defend Euthyphro’s suggestion that a devout act is one that God loves? With this question, analyze carefully the issues surrounding divine command theory, including Leibniz’s position toward it.
% Euthyphro suggests that a devout act is one that is loved by the gods, and so 
% I don't like this quention, because it makes silly assumptions that I would have to argue within.
%2) Are Socrates’ reasons for not escaping as they are given in the Crito specific to his situation or are they valid for people generally? Do not neglect the details of Socrates’ situation.
%3) Must we always obey the law? If not, under which conditions may we be justified in transgressing it? Take into account the Crito and the Apology.
%4) If ordered by the authorities of our country to do something, should we always obey, if not when might we be justified in disobeying? As with question 3, take into account the Crito and the Apology.
%5) Socrates’ defense in the Apology, a success or a failure? Issues discussed in the Crito may assist in addressing this topic.
%6) According to Plato, how does one acquire new knowledge and how does this relate to the Socratic dialectic?  
% This question seems like the easiest one to answer, as Meno and Euthyphro are pretty easy reads.
%7) Is Leibniz’s solution to the problem of evil adequate? With this topic you wish to keep in mind his opening assumption.
% If a world that necessitates evil in order to derive a greater good is considered "perfect" in God's eyes, then what is our relationship to God? If God is perfectly cappable of creating a world without evil, in his omnipotence, then why doesn't he do so? If God is indeed perfectly benevolent in everything He does, then either God must consider what we percieve as evil in the world to be benevolent, or God would rather ignore some ammount of evil to achieve some greater good that his omnipotence was somehow unable to achieve. Both of these hypothesis lead to interesting trains of thought which I will breifly explore.
%
% Okay, here's the plan:
% Thesis: Leibniz's solution for the problem of evil is not adequate, because the assumptions he makes concerning the perfection of God introduce difficulties that he does not wholly address.
% Introduction: Leibniz's notion of perfection has three parts, omnipotence, omniscience, and perfect morallity. He needs to justify these three perfections in his arguments.
% First Paragraph: Assume God is all knowing and benevolent. Why does God need to use evil to derive good, if he is also all powerful?
%   If God needs proportionate ammounts of both good and evil, and God is indeed infinite, if we take the limit of this good and evil we end up with an amoral (amoral != evil) continuum of both good and evil, making God's benevolence a moot point.
% Third Paragraph: Assume God is all knowing and all powerful. How can we define God's 

At the beginning of his \textit{Discorse on Metaphysics}, Leibniz establishes the assumption that God is perfect and does everything in the most desireable way. He defines this perfection as God having both omnipotence and omniscience, stating, ``power and knowledge are perfections, and, insofar as they belong to God, they do not have limits." He extends this assumption of perfection to morality, adding, ``God, possessing supreme and infinite wisdom, acts in the most perfect manner, not only metaphysically, but also morally speaking." Througought the rest of Leibniz's discorse, the validity of Leibniz's agruments rests on his ability to satisfy these three perfections of God without leading to any contradictions. It is quite easy to justfify God's seemingly imperfect actions by disregarding one or more of these three assumptions of perfection. Leibniz never entertains the notion of an imperfect God for very long, and his explenation for the existance of evil in the world is no exception. The validity of his solution for the problem of evil is dependant on how well he can justify these three assumptions of God's perfection at the same time.
% Directly after introducing his notion of a perfect god, Leibniz takes the time to challenge some common opinions which stand contrary to these assumptions, namely that God's notion of goodness is arbitrary and that God did not make a perfect world.



The question of the morality of God is much more difficult than that of omnipotence or omniscience in that morality is difficult to define. A clear definition of morality or good is something I believe that Leibniz doesn't adequately address in this discourse, possibly because that's an old philosophical problem that philosophers and theologians have struggled with since before the time of Socrates and Plato. Leibniz admits these difficulties, and comes close to explaining how God determines what is good in his discussion on final causes:
\begin{mlaquote}
It is a great mistake to believe that God made the world only for us, although it is quite true that he made it in its entirety for us and that there is nothing in the universe which does not affect us and does not also accommodate itself in accordance with his regard for us, following the principles set forth above. Thus when we see some good effect or perfection occuring or ensuing from God's works, we can say with certainty that God has proposed it.
\end{mlaquote}
In this quote, Leibniz is arguing against the same ``new philosophers" that he refuted in the beginning of his discourse. Leibniz seems to define good as God creating everything in the universe to affect us ``with his regard for us." Leibniz is making the assumption that God made the universe for man, and that everything in it is made with man's best interests in mind. This is a pleasent and reassuring assumption at its face, but it doesn't take very much reflection to see that there is much in this world that man detests and causes him harm. Voltaire's satirical novel \textit{Candide} demonstrates this effectively, without needing to go into a lofty discussion about metaphysics.

Leibniz's solution for the problem of evil is not adequate, because the assumptions he makes concerning the perfection of God introduce difficulties that he does not wholly address. If we assume that God is moral and all knowing, difficulties arise when  If we expect God to be perfect in knowledge and ability, difficulties arise when we 


%\begin{workscited}
%
%\bibent
%Doe, John.  ``Paper Title."  \textit{Book Title}.  4 July 1776.
%
%\bibent
%Doe, John.  ``Paper Title."  \textit{Book Title}.  4 July 1776.
%
%\end{workscited}
\end{mla}
\end{document}
