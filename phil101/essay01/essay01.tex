\documentclass[12pt,letterpaper]{article}
\usepackage{ifpdf}
\usepackage{mla}
\begin{document}
\begin{mla}{Jonathan}{Glines}{Professor Pelletti}{Philosophy 1101}{28 September 2011}{Justifying Evil in a Perfect World on Gottfried Leibniz's \textit{Discourse on Metaphysics}}

%1) Would Monotheistic beliefs allow one to defend Euthyphro’s suggestion that a devout act is one that God loves? With this question, analyze carefully the issues surrounding divine command theory, including Leibniz’s position toward it.
% Euthyphro suggests that a devout act is one that is loved by the gods, and so 
% I don't like this quention, because it makes silly assumptions that I would have to argue within.
%2) Are Socrates’ reasons for not escaping as they are given in the Crito specific to his situation or are they valid for people generally? Do not neglect the details of Socrates’ situation.
%3) Must we always obey the law? If not, under which conditions may we be justified in transgressing it? Take into account the Crito and the Apology.
%4) If ordered by the authorities of our country to do something, should we always obey, if not when might we be justified in disobeying? As with question 3, take into account the Crito and the Apology.
%5) Socrates’ defense in the Apology, a success or a failure? Issues discussed in the Crito may assist in addressing this topic.
%6) According to Plato, how does one acquire new knowledge and how does this relate to the Socratic dialectic?  
% This question seems like the easiest one to answer, as Meno and Euthyphro are pretty easy reads.
%7) Is Leibniz’s solution to the problem of evil adequate? With this topic you wish to keep in mind his opening assumption.
% If a world that necessitates evil in order to derive a greater good is considered "perfect" in God's eyes, then what is our relationship to God? If God is perfectly cappable of creating a world without evil, in his omnipotence, then why doesn't he do so? If God is indeed perfectly benevolent in everything He does, then either God must consider what we percieve as evil in the world to be benevolent, or God would rather ignore some ammount of evil to achieve some greater good that his omnipotence was somehow unable to achieve. Both of these hypothesis lead to interesting trains of thought which I will breifly explore.
%
% Okay, here's the plan:
% Thesis: Leibniz's solution for the problem of evil is not adequate, because the assumptions he makes concerning the perfection of God introduce difficulties that he does not wholly address.
% Introduction: Leibniz's notion of perfection has three parts, omnipotence, omniscience, and perfect morallity. He needs to justify these three perfections in his arguments.
% First Paragraph: Assume God is all knowing and benevolent. Why does God need to use evil to derive good, if he is also all powerful?
%   If God needs proportionate ammounts of both good and evil, and God is indeed infinite, if we take the limit of this good and evil we end up with an amoral (amoral != evil) continuum of both good and evil, making God's benevolence a moot point.
% Third Paragraph: Assume God is all knowing and all powerful. How can we define God's 

At the beginning of his \textit{Discourse on Metaphysics}, Leibniz establishes the assumption that God is perfect and does everything in the most desirable way. He defines this perfection as God having both omnipotence and omniscience, stating, ``power and knowledge are perfections, and, insofar as they belong to God, they do not have limits" (\textit{Discourse on Metaphysics} 1.2). He extends this assumption of perfection to morality, adding, ``God, possessing supreme and infinite wisdom, acts in the most perfect manner, not only metaphysically, but also morally speaking" (\textit{Discourse on Metaphysics} 1.3). Throughout the rest of Leibniz's discourse, the effectiveness of Leibniz's arguments rests on his ability to satisfy these three perfections of God without leading to any contradictions. It is quite easy to justify God's seemingly imperfect actions by disregarding one or more of these three assumptions of perfection. Leibniz never entertains the notion of an imperfect God for very long, and his explanation for the existence of evil in the world is no exception. The validity of his solution for the problem of evil is dependent on how well he can justify these three assumptions of God's perfection at the same time.
% Directly after introducing his notion of a perfect god, Leibniz takes the time to challenge some common opinions which stand contrary to these assumptions, namely that God's notion of goodness is arbitrary and that God did not make a perfect world.

Leibniz addresses the problem of evil directly in his discussion on miracles and the general order. He explains, regarding actions that happen to be evil, ``The course of things corrects [the action's] evilness and repays the evil with interest in such a a way that in the end there is more perfection in the whole sequence than if the evil had not occurred" (\textit{Discourse on Metaphysics} 7.3). He concludes that, while God does not will the evil action, he knows how to ``draw a greater good from it" (\textit{Discourse on Metaphysics} 7.3). This explanation is difficult to accept, for a couple of reasons. Firstly, in other parts of the discourse, Leibniz contends that God wills the most perfect world, which would imply that God wills the evil along with the beneficial final cause of that evil. Leibniz goes to great lengths to argue that God can will perfection (the one and only perfect world) and at the same time allow for evil and free will, but the dubiousness of this claim remains. Secondly, if God is omnipotent, it does not follow that he would have to rely upon something such as evil in order to achieve his goals. Using evil as a means to a better end is an unfortunate character trait of many imperfect humans, a trait that hardly anyone would consider moral. This is a clear contradiction between God's assumed morality, and his supposed omnipotence.

The question of the morality of God is much more difficult than that of his omnipotence or omniscience in that morality is difficult to define. A clear definition of morality or the distinction between good and evil is something I believe that Leibniz does not adequately address in this discourse, possibly because this is an old philosophical problem that philosophers and theologians have struggled with since before the time of Socrates or Plato. Leibniz admits these difficulties, and comes close to explaining how God determines what is good in his discussion on final causes:
\begin{mlaquote}
It is a great mistake to believe that God made the world only for us, although it is quite true that he made it in its entirety for us and that there is nothing in the universe which does not affect us and does not also accommodate itself in accordance with his regard for us, following the principles set forth above. Thus when we see some good effect or perfection occurring or ensuing from God's works, we can say with certainty that God has proposed it. (\textit{Discourse on Metaphysics} 19.2)
\end{mlaquote}
Leibniz is making the assumption that God made the universe for man, and that everything in it is made with man's best interests in mind. This is a pleasant and reassuring assumption at its face, but it doesn't take very much reflection to see that there is much in this world that man detests and causes him harm. Voltaire's satirical novel \textit{Candide} demonstrates this effectively, without needing to go into a lofty discussion about metaphysics.

A surprisingly common and interesting solution for the problem of God's morality is that God alone defines what is moral, independent of the intrinsic qualities of things that we might consider to make something moral or good. This solution, known as divine command theory, is much more logically consistent than Leibniz's solution, because what is moral with regards to man is a subject of debate that we will likely never be able to define, while it is reasonable that a singular god could come to a single definition of morality without any disagreement with himself. This same moral dilemma appeared in Plato's \textit{Euthyphro} when Socrates tries to determine what the gods consider pious. This dilemma is more or less avoided by having a single deity, which might explain the popularity of monotheism in modern religions. However consistent, the implication for this notion is that what God considers moral is essentially independent of what we as humans consider moral, and that God must be an all-powerful, amoral being whom we should fear and obey. Note the distinction between the words amoral and evil, because the human concepts of both good and of evil would be independent of those of God. Leibniz refutes this at the beginning of his discourse, stating that it would ``destroy all of God's love and all of his glory" (\textit{Discourse on Metaphysics}). Apparently, God's love for man is another assumption that Leibniz takes to heart, which is a perfectly understandable assumption for a deeply religious person to make. However, in light of the many harsh realities of the world, it seems that Leibniz has all too quickly dismissed a simpler and more defensible solution in favor of a solution that is plagued with subjectivity and contradictions.

Leibniz's solution for the problem of evil is not adequate, because the assumptions he makes concerning the perfection of God introduce difficulties that he does not wholly address. If we expect that God is moral and all knowing, difficulties arise when we try to establish the omnipotence of a God that tries to achieve good ends through evil. Likewise, even if we assume God to be perfect in knowledge and ability, there are clear examples of violations of his perfect morality, provided that we define that which benefits man to be moral. In the end, defining what is good and evil and what constitutes God's morality in the first place is very problematic, as his morality doesn't seem to be aligned with that of most people. The assumptions that Leibniz makes on God's perfection do not ring true throughout his discourse, and this is what cripples his solution for evil persisting in the best of possible worlds.


\begin{workscited}

\bibent
Voltaire.  \textit{Candide}.  Trans. Philip Littell.  New York: Boni and Liverlight, Inc., 1918.  Print.

\end{workscited}
\end{mla}
\end{document}
